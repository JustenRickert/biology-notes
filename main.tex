% Created 2017-04-12 Wed 18:51
% Intended LaTeX compiler: pdflatex
\documentclass[11pt]{article}
\usepackage[utf8]{inputenc}
\usepackage[T1]{fontenc}
\usepackage{graphicx}
\usepackage{grffile}
\usepackage{longtable}
\usepackage{wrapfig}
\usepackage{rotating}
\usepackage[normalem]{ulem}
\usepackage{amsmath}
\usepackage{textcomp}
\usepackage{amssymb}
\usepackage{capt-of}
\usepackage{hyperref}
\usepackage[margin=1in]{geometry}
\usepackage{lmodern}
\author{Justen Rickert}
\date{\today}
\title{biology 1010}
\hypersetup{
 pdfauthor={Justen Rickert},
 pdftitle={biology 1010},
 pdfkeywords={},
 pdfsubject={},
 pdfcreator={Emacs 26.0.50.1 (Org mode 9.0.5)}, 
 pdflang={English}}
\begin{document}

\maketitle
\tableofcontents


\section{packet}
\label{sec:org83b50b7}
\subsection{Classification of Major Biochemicals}
\label{sec:orge751b41}
1.\hyperref[sec:orga968f94]{Carbohydrates}
\begin{enumerate}
\item \hyperref[sec:org9883b36]{Monosaccharides}
\begin{itemize}
\item Deoxyribose
\item glucose
\item fructose
\end{itemize}
\item \hyperref[sec:org5773569]{Disaccharides}
\begin{itemize}
\item sucrose
\item lactose
\end{itemize}
\item \hyperref[sec:org605872d]{Polysaccharides}
\begin{itemize}
\item glycogen
\item starch
\item cellulose
\end{itemize}
\end{enumerate}
\begin{enumerate}
\item \hyperref[sec:orged5276d]{Lipids}
\begin{enumerate}
\item \hyperref[sec:org9f7d610]{Fatty acids}
\item \hyperref[sec:orgb206f86]{Phospholipids}
\item \hyperref[sec:org2f5773c]{Triglycerides}
\item \hyperref[sec:org1f6dd99]{Steroids}
\end{enumerate}
\item \hyperref[sec:org01ca357]{Proteins} (Polymers made of \emph{amino acid} monomers)
\begin{itemize}
\item \hyperref[sec:orgb977ee1]{Gluten}
\item \hyperref[sec:org9d0c5b8]{Insulin}
\item \hyperref[sec:orgf2c3b90]{Elastin}
\item \hyperref[sec:org7fddd30]{Collagen}
\item \hyperref[sec:org79e368e]{Hemoglobin}
\end{itemize}

\item Nucleic Acids
\begin{itemize}
\item \hyperref[sec:org9458e10]{DNA}
\item \hyperref[sec:org55a0135]{RNA}
\end{itemize}
\end{enumerate}

\subsubsection{Carbohydrates \emph{macronutrient}}
\label{sec:orga968f94}
sugars, starches, and fibers found in fruits, grains, vegetables, and milk
products. One of three main ways the body obtains energy or calories.

\emph{biological molecule} consisting of carbon (C), hydrogen(H), and oxygen(O).

\begin{enumerate}
\item Monosaccharides
\label{sec:org9883b36}
\begin{itemize}
\item pop has monosaccharides
\item fructose is sweet
\begin{itemize}
\item When there is no exercise, fructose \(\rightarrow\) glycogen \(\rightarrow\) fat.
\end{itemize}

\item Latin \emph{mono-} one, smallest
\end{itemize}

\begin{enumerate}
\item Deoxyribose [2-deoxyribose] C\(_{\text{5}}\)H\(_{\text{10}}\)O\(_{\text{4}}\)
\label{sec:orgcb10a54}
\begin{itemize}
\item Modern \emph{deoxy sugar} less oxygen; it is derived from the sugar ribose.
\end{itemize}

\item Glucose C\(_{\text{6}}\)H\(_{\text{12}}\)O\(_{\text{6}}\)
\label{sec:orgd62c86c}
sugar with 6 carbon atoms classed as monosaccharides

circulates in the blood of animals as blood sugar. In plants is made
during photosynthesis from water and carbon dioxide, using energy from
sunlight.
\end{enumerate}

\item Disaccharides
\label{sec:org5773569}
\begin{itemize}
\item Latin \emph{di-} two
\end{itemize}

\begin{enumerate}
\item sucrose
\label{sec:orgca32215}
Sucrose is found in many food plants.

\item maltose
\label{sec:org253185b}
Produced during the malting of barley.
\end{enumerate}

\item Polysaccharides
\label{sec:org605872d}
\emph{polymeric} \emph{carbohydrate} molecules composed of long chains of monosaccharides

\begin{itemize}
\item Latin \emph{poly-} many
\end{itemize}

\begin{enumerate}
\item Glycogen
\label{sec:org6f84651}

multibranched \emph{polysaccharide} of \emph{glucose}. Main storage form of
\emph{glucose} in the body.

\item Starch
\label{sec:org613d2b8}
polymeric carbohydrate with a large number of glucose joined by glycosidic
bonds. Produced by most green plants as an energy store. Most common
carbohydrate in human diets.

\begin{description}
\item[{glycosidic bonds}] covalent bond that joins a carbohydrate (sugar)
molecule to another group, which may or may not be another
carbohydrate.
\end{description}

\item Cellulose C\(_{\text{6}}\)H\(_{\text{10}}\)O\(_{\text{5}}\)
\label{sec:orgf6a7c6d}
organic compund, a linear chain of several hundred to many thousands of
\(\beta\)(1\(\rightarrow\)4) linked D-glucose units. Important structural component of the
primary cell wall of green plants, many forms of algae, and \emph{oomycetes}.

\begin{description}
\item[{D-glucose}] D-isomer of \emph{glucose}, also known as \emph{dextrose}, occurs widely
in nature.

\item[{Oomycetes}] distinct phylogenetic lineage of fungus-like eukaryotic
microorganisms.
\end{description}
\end{enumerate}
\end{enumerate}

\subsubsection{Lipids}
\label{sec:orged5276d}
They are basically fats.

\begin{enumerate}
\item Fatty acids
\label{sec:org9f7d610}
Carboxylic acid with a long aliphatic chain. This guy looks crazy, though.

The chains range from 4 to 28 carbon atoms.

\begin{itemize}
\item Unsaturated fatty acids
\begin{itemize}
\item cis
\item trans
\end{itemize}
\end{itemize}

\item Phospholipids
\label{sec:orgb206f86}
Major component of all cell membranes. They can form \emph{lipid bilayers}
because of their \emph{amphiphilic} characteristic. (It has \emph{hydrophilic} water
loving and \emph{lipophilic} fat loving properties.) 

\emph{hydrophilic phosphate} head, \emph{hydrophobic fatty acid} tails

\begin{itemize}
\item \emph{amphis} both \emph{philia} love, friendship \emph{hydro-} water
\end{itemize}

\item Triglycerides
\label{sec:org2f5773c}
Ester derived from \emph{glycerol} and three \emph{fatty acids}. Having a high amount
of \emph{triglycerides} can be very bad for your heart. Triglycerides are
\emph{lipids} and are fatty.

\begin{enumerate}
\item glycerol
\label{sec:org9722b49}
Colorless, odorless, viscous liquid; sweet, non-toxic. This is the
backbone found in all lipids know as triglycerides.

Used in food industry as a sweetener and humectant and in pharmaceutical
formulations.

\begin{description}
\item[{humectant}] keeps things moist.
\end{description}
\end{enumerate}

\item Steroids
\label{sec:org1f6dd99}
Organic compound with four rings arranged in a specific configuration.
Steroids serve two principal biological functions: certain steroids are
important components of cell membranes which alter membrane fluidity, and
many steroids are signaling molecules which activate steroid hormone
receptors.

\begin{enumerate}
\item sex hormones
\label{sec:orgc3c2194}
Influence sex differences and support reproduction. 

\begin{itemize}
\item \emph{androgens}, \emph{estrogens}, and \emph{progestagens}
\end{itemize}

\item corticosteroids
\label{sec:org0c969ff}
With natural product classes the \emph{glucocorticoids} and the
\emph{mineralocorticoids}

\begin{description}
\item[{glucocorticoids}] Regulates many aspects of metabolism and immune function.

\item[{mineralocorticoids}] Help maintain blood volume and control renal
excretion of electrolytes.
\end{description}
\end{enumerate}
\end{enumerate}

\subsubsection{Proteins}
\label{sec:org01ca357}
Large biomolecules, or macromolecules consisting of one or more long chains
of \emph{amino acid} \emph{residues}. Just like \emph{polysaccharides} and \emph{nucleic acids},
\emph{proteins} are essential parts of organisms and participate in virtually
every process within cells.

They have an average lifespan of 1-2 days in mammalian cells.

\begin{enumerate}
\item Gluten
\label{sec:orgb977ee1}
Mixture of proteins found it wheat and related grains.

\begin{itemize}
\item Latin \emph{gluten} glue
\end{itemize}

\item Insulin
\label{sec:org9d0c5b8}
Peptide hormone produced by beta cells. Affects the metabolism of
carbohydrates.

\item Elastin
\label{sec:orgf2c3b90}
Highly elastic protein in connective tissue. Elastin helps skin to return
to its original position when it is poked or pinched.

\item Collagen
\label{sec:org7fddd30}
Main structural protein in the extracellular space in the various
connective tissues in animal bodies.

\item Hemoglobin (Hb `or` Hgb)
\label{sec:org79e368e}
Iron-containing oxygen-transport metalloprotein. Hemoglobin in the blood
carries oxygen from the respirator organs to the rest of the body.

Also haemoglobin

\begin{description}
\item[{metalloprotein}] generic term for a protein that contains a \emph{metal ion
cofactor}.
\end{description}
\end{enumerate}

\subsubsection{Nucleic Acids}
\label{sec:org61c554e}
\begin{enumerate}
\item Deoxyribonucleic acid
\label{sec:org9458e10}
Molecule that carries the genetic instructions used in the growth,
development, functioning, and reproduction of all know living organisms and
many viruses.

\item Ribonucleic acid
\label{sec:org55a0135}
Polymeric molecule essential in various biological roles in coding,
decoding, regulation, and expression of genes.
\end{enumerate}

\section{book}
\label{sec:org5d99041}
\subsection{Ch. 1}
\label{sec:orgb0178b0}
\subsubsection{Seven characteristics of life}
\label{sec:orge9955b2}
\begin{itemize}
\item being made of cells

Single-cell organisms have everything they need to be self-sufficient. In
multicellular organisms, specialization increases until some cells do only
certain things.

\item responsiveness to the environment

Living things will make changes in response to a stimulus in their
environment. A behavior is a complex set of responses.

\item growth and change

Cell division -- the orderly formation of new cells. cell enlargement --
the increase in size of a cell. An organisms gets larger as the number of
its cells increases.

\item ability to reproduce

Reproduction is not essential for survival of individual organisms, but
must occur for species to survive.

\item have the metabolism and breathe / have energy

Living things take in energy and use it for maintenance and growth.

\item maintain homeostasis

There is an organization to all the parts of life that means they all
depend on each other, and they must all be in equilibrium.

\item passing traits onto offspring

Adaptions are traits giving an organism an advantage in a certain
environment.
\end{itemize}

\begin{enumerate}
\item Macromolecules
\label{sec:orgdb1dc48}

Four major molecules essential for all known forms of life are 

\begin{itemize}
\item RNA
\item DNA
\item Proteins
\item Carbohydrates
\end{itemize}

\item Enzyme
\label{sec:org8e375a1}

Macromolecule biological catalyst. \emph{Enzymes} accelerate, or catalyze,
chemical reactions. Lower \emph{activation energy} to increase rate of a
reaction. 

Usually \emph{proteins}. A lot of the time used to break down more complex
chemicals and compounds.
\end{enumerate}

\subsubsection{Scientific Method}
\label{sec:org9998bb7}
\begin{itemize}
\item Ask a Question
\item Construct a Hypothesis
\item Test with experiment
\item Analyze data and draw conclusions
\item Communicate results
\end{itemize}

\subsection{Ch. 2}
\label{sec:orge0380f5}
Scale for acidity or basicity of an aqueous solution. Measure of the hydrogen
ion density in the the solution.

\begin{itemize}
\item Carbon is the basis of organic molecules.

\item Lipids are not soluble in water.

\item[{ATP}] \emph{Adenosine triphosphate} A small molecule used in cells as a
coenzyme.

The molecular unit of currency of intracellular energy transfer.
\end{itemize}

\subsection{Ch. 3}
\label{sec:org2aa29a1}
\begin{description}
\item[{Cholesterol}] An organic molecule; a sterol (or modified steroid), a lipid
molecule and is biosynthesized by all animal cells. An essential
structural component of all animal cell membranes that is required to
maintain both membrane structural integrity and fluidity.

\begin{itemize}
\item Latin \emph{chole-} bile \emph{stereos} solid
\end{itemize}

\item[{Fluid mosaic}] The plasma membrane can be described as fluid combination
of phospholipids, cholesterol, and proteins. Fluid on inside and outside
of the membrane.

\item[{Diffusion}] Movement of molecules or atoms from a region of high
concentration to a region of low concentration.

\item[{Osmosis}] Movement of solvent molecules \textbf{through a semi-permeable
membrane} into a region of higher solute concentration.
\end{description}

\subsubsection{Diffusion vs. Active Transport}
\label{sec:org02d679d}

\begin{center}
\begin{tabular}{ll}
Diffusion (Passive Transport) & Active Transport\\
\hline
quick & slow\\
down a gradient & up a gradient\\
no energy & need energy\\
no protein & need protein\\
\end{tabular}
\end{center}

\begin{description}
\item[{Endocytosis}] Form of active transport in which a cell transports
molecules by engulfing them in an energy-using process.

\item[{Exocytosis}] Form of active transport in which cell transports molecules
out of the cell.
\end{description}

\subsection{Ch. 5}
\label{sec:orgc7fec7d}
The \emph{skeletal system} is composed of bones, ligaments, and cartilage; it is an
\emph{organ system}. It \uline{supports} and \uline{protects} the other organ systems of the
body and provides a structure that enables \uline{movement}.

Bones are living cells surrounded by extracellular deposits of calcium
minerals. 
\begin{itemize}
\item Support
\item Movement
\item Protection
\item Formation of blood cells
\item Mineral and fat storage
\begin{itemize}
\item like phosphorous, calcium
\end{itemize}
\end{itemize}

\subsubsection{four types of bones}
\label{sec:orge7da9fc}
\begin{description}
\item[{long}] arms, legs
\item[{short}] wrist, feet
\item[{flat}] ribs, skull
\item[{regular}] all the rest
\end{description}

\subsubsection{Connective tissues}
\label{sec:orgb9b043b}
\begin{description}
\item[{Bone}] Hard elements of the skeleton. Store minerals and produce the
cellular components of blood.

\emph{Bones} are hard because of nonliving extracellular crystals of calcium
 minerals; hard, rigid, appearance and feel. \emph{Bones} are living cells,
 and undergo constant replacement, remodeling, and repair.
\begin{itemize}
\item support
\item protection
\item movement
\end{itemize}
\item[{Long bone}] It is longer than it is wide, consists of a cylindrical
shaft \emph{\textbf{diathesis}}, and an enlarged knob called an \emph{\textbf{epiphysis}}.
\item[{Compact bone}] Forms the shaft and covers each end.
\begin{description}
\item[{Osteocytes}] Mature bone cells that maintain the structure of bone.
The \emph{compact bone} is made up largely of extracellular deposits of
calcium phosphate enclosing and surrounding living cells.
\begin{itemize}
\item Greek \emph{osteo} bone \emph{cyte} cell
\end{itemize}
\item[{Osteons}] [Haversian systems] Osteocytes arranged in rings in
cylindrical structures.
\item[{Central canal}] [Haversian canal] \emph{Ostecytes} nearest the center of an
\emph{osteon} receive nutrients from diffusion from blood vessels.

\item[{Spongy bone}] Inside each epiphysis. It is a latticwork of hard,
relatively string \emph{trabeculae} (Latin "little beams") composed of
calcium minerals and living cells.

\emph{\textbf{note}} Spongy bone doesn't need to rely on central canals for
nutrients and waste removal. The slender \emph{trabecular} structure of
spongy bone gives each osteocyte access to nearby blood vessels in red
bone marrow.

A \emph{trabecular} structure is a latticework of interconnected beams.
\end{description}

\item[{Joint}] The point of contact between bones.

\item[{Ligaments}] Dense, fibrous connective tissue that attach bone to bone.
They are a regular array of closely packed collaen fibers all oriented
in the same direction, with just a few \emph{fibroblasts} in between.

\emph{Ligaments} are slow to heal because they have very few living cells and
a poor blood supply.

\item[{Cartilage}] Specialized connective tissue consisting primarily of fibers
of collagen and elastic in a gel-like fluid called \emph{ground substance}.

\begin{description}
\item[{Fibrocartilage}] Collagen fibers arranged in thick bundles.
\item[{Hyaline cartilage}] Smooth, glassy cartilage of thin collagen fibers.
\item[{Elastic cartilage}] Mostly elastin fibers, so it is highly flexible.
\end{description}
\end{description}

\subsubsection{Bone development}
\label{sec:orge9d0d17}
At the earliest stages of fetal development, before organs develop, we have
\emph{chondroblasts}. After about two to three months of fetal development,
\emph{ossification} happens.

\begin{description}
\item[{Chondroblasts}] Cartilage-forming cells that build a model of the future
bone. These \emph{chrondroblasts} are rudimentary models of future bones and
are created out of \emph{hyaline cartilage}.

\item[{Ossification}] Chondroblasts slowly die out and the cartilage models
begin to dissolve and are replaced by bone.

\item[{Osteoblasts}] \uline{Young bone-forming} cells that cause the hard extracellular
matrix of bone to develop.

They secrete both \emph{osteoids}, and \emph{hydroxyapatite}. In \uline{mature} \emph{compact bone},
approximately one-third of the structure is \emph{osteoid} and two-thirds is
crystals of \emph{hydroxyapatite}.

Eventually, the rate at which \emph{osteoblasts} produce the \emph{osteoid}
matrix and stimulate the mineral deposits declines, and \emph{\textbf{osteoblasts}}
\textbf{become} mature \emph{\textbf{osteocytes}} embedded in their individual \emph{lacunae}.

\begin{description}
\item[{Osteoid}] Mixture of proteins that forms a matrix that provides
internal structure and strength to bone.

\item[{Hydroxyapatite}] Hard mineral salts of calcium phosphate
\end{description}

\item[{Osteclasts}] Bone-dissolving cells.

Cut through mature bone tissue, dissolving the \emph{hydroxyapatite} and
digesting the \emph{osteoid} matrix in their path. The released calcium and
phosphate ions enter the blood.

The areas from which bone has been removed attract new osteoblasts,
which lay down new osteoid matrixes and stimulate the deposition of new
hydroxyapatite crystals.

\item[{Growth plate}] [epiphyseal plate] A narrow stip of \emph{cartilage} that
remains in each \emph{epiphysis}. This is where there is bone on one side,
and cartilage on the other side.

\item[{Osteoporosis}] Common bone disease in which bones lose a great deal of
bone mass (seemingly becoming "porous"). The honeycomb shape of bone
becomes more porous for those with \emph{osteoporosis}.

\item[{Hematoma}] Mass of clotted blood

\item[{Fibrocarttilage callus}] When a bone breaks, this forms between the
broken ends and is later replaced with bone.

\item[{Parathyroid hormone}] [PTH] stimulates the \emph{osteoclasts} to secrete more
bone-dissolving enzymes.

\item[{Calcitonin}] stimulates \emph{osteoblast} activity, causing calcium and
phosphate to be removed from blood and deposited in bone.
\end{description}

\subsubsection{Skeletal system}
\label{sec:orgf6f87dc}
The \emph{skull} and \emph{vertebral column} protect the \emph{brain} and \emph{spinal cord},
the \emph{rib cage} protects the organs of the chest cavity, and the \emph{pelvic
girdle} supports the body's weight and protects the pelvic organs. The upper
limbs are capable of a wide range of motions (dexterous movement). The lower
limbs are stronger but less dexterous than the upper limbs
\begin{description}
\item[{Axial skeleton}] 80 bones, including the skull, vertebral column, ribs,
and sternum.
\begin{description}
\item[{Cranium}] [skull] Comprises over two dozen bones that protect the brain and
form the structure of the face.
\item[{Cranial bones}] Flat bones in the skull that enclose and protect the brain.
\begin{description}
\item[{Frontal bone}] Comprises the forehead and the upper ridges of the eye
sockets.
\item[{Parietal bones}] [two] Upper left and right sides of the skull.
\item[{Temporal bones}] [two] Forms the lower left and right sides of the skull.
\item[{Sphenoid bone}] Forms the back of both eye sockets.
\item[{Ethmoid bone}] Contributes to the eye sockets and also helps support the
nose.
\item[{Nasal bones}] [two] Small, narrow; underlie only the upper bridge of the
nose; the rest of the fleshy protuberance called the nose is made up of
cartilage and other connective tissue.
\item[{Lacrimal bones}] [small] At the inner eye socket, are pierced by a tiny
opening through which the tear ducts drain tears from the eye sockets
into the \emph{nasal cavity}.
\item[{Mandible}] [lower jaw] Contains the sockets that house the lower row of
teeth.
\item[{Occipital bone}] Curving underneath to form the back and base of the
\emph{skull}.
\item[{Foramen magnum}] [Latin "great opening"] Where the \emph{vertebral column}
connects to the \emph{skull} and the \emph{spinal cord} enters the \emph{skull} to
communicate with the brain.
\end{description}
\item[{Facial bones}] compose the front of the skull.
\begin{description}
\item[{Maxilla}] On either side of the nose, forming part of the eye sockets and
contain the sockets that anchor the upper row of teeth.
\item[{Palatine bones}] [two] Irregular bones, comprise the hard palate at
the back part of the nasal cavity.
\item[{Zygomatic bones}] [cheekbone or malar bone] paired bone which
articulates with the maxilla.
\end{description}
\item[{Sinuses}] Air spaces in the \emph{cranial} and \emph{facial bones} that make the
skull lighter and give the human voice its characteristic tone and
resonance.

Each sinus secretes \emph{mucus}, a thick, sticky fluid that helps trap
foreign particles in incoming air. They connect to the \emph{nasal cavity}
via small passageways through which the mucus normally drains.
\item[{Hyoid bone}] Does not make direct contact with other bones of the
\emph{axial skeleton}; it is attached to the \emph{temporal bone} only by
\emph{ligaments}. Serves as a point of attachment for muscles of the
\emph{tongue}, the \emph{larynx}, and the \emph{pharynx}.
\item[{Vertebral column}] The body's main axis. It \uline{supports} the head,
\uline{protects} the \emph{spinal cord}, and serves as the site of \uline{attachment}
for the four limbs and various muscles. It has 5 anatomical regions.

Neighboring vertebrae are separated from each other by a flat,
elastic, compressible \emph{\textbf{intervertebral disk}} composed of a soft,
gelatinous center and a tough outer layer of \emph{fibrocartilage}.
\begin{description}
\item[{Cervical vertebrae}] [7 vertebrae] Neck
\item[{Thoracic vertebrae}] [12 vertebrae] The chest or thorax
\item[{Lumbar vertebrae}] [5 vertebrae] Lower portion or "small" of the back
which forms the lumbar curve of the spine
\item[{Sacral}] [5 vertebrae fused] In the \emph{sacrum} or upper pelvic region
\item[{Coccygeal}] [4 vertebrae fused] The \emph{coccyx} or tailbone. It is an
example of a \emph{vestigial structure}.
\end{description}
\item[{Ribs}] One end of each rib branches from the \emph{thoracic} region of the
\emph{vertebral column}. The other ends of the upper seven pairs attach
via \emph{cartilage} to the \emph{sternum}.

Humans have 12 pairs of ribs.
\item[{Sternum}] [breastbone] A flat blade-shaped bone composed of three
separate bones that fuse during development.
\end{description}
\item[{Appendicular skeleton}] \{Pectoral girdle, pelvic girdle, and limbs\}
Consists of those parts of the body that attach, or are appended to,
the \emph{axial skeleton}. Includes the arms, legs, and their attachments to
the trunk, which are the \emph{pectoral} and \emph{pelvic girdles}.
\begin{description}
\item[{Pectoral girdle}] Lends flexibility to the upper limbs. 
\begin{description}
\item[{Clavicle}] [collar bone] Extend across the top of the chest and
attach to the \emph{scapulas}.

Most frequently broken bone.
\item[{Scapula}] [shoulder blade] The triangular bones in the upper back.
\end{description}
\item[{arm and hand}] Consists of 30 different bones.

The humerus is the first and meets at the scapula. The lower end of
the \emph{radius} an \emph{ulna} meet the \emph{carpal bones}. A group of eight
small bones that make up the wrist. The five \emph{metacarpal} bones form
the palm of the hand, and they join with the 14 \emph{phalanges}, which
form the fingers and thumb.
\begin{description}
\item[{Humerus}] [upper arm] The long bone of the upper arm, fits into a socket in the
\emph{scapula}.
\item[{Ulna}] Meets with the other end of the humerus as does the \emph{radius}.
Bone of the forearm.

The elbow around here is nicknamed the "funny bone."
\item[{Radius}] bone of the forearm.
\item[{Carpal}] [wrist]

Repetitive motions can lead to health problems called \emph{repetitive
stress syndromes}. One well-known repetitive stress syndrome is
\emph{carpal tunnel syndrome}. Inflammation of the tendons causes them
to press against the nerve supplying the hand, resulting in pain,
tingling, or numbness in the wrist and hand.
\item[{Metacarpals}] [hand]
\item[{Phalanges}] [finger bones]
\end{description}
\item[{Pelvic girdle}] Supports the body, and consists of two \emph{coxal} bones
and the \emph{sacrum} and \emph{coccyx} of the \emph{vertebral column}. 

The primary function of the pelvic girdle is to support the weight of
the upper body against the force of gravity. The lower limbs are
larger and more firmly connected to the rest of the body than the
pectoral girdle and upper limbs.
\begin{description}
\item[{Coxal bone}] Attach to the \emph{sacral} region of the \emph{vertebral column} in
back, then curve forward to meet in from at the \emph{pubic symphysis},
which is right around the groin it looks like, where they are joined
by cartilage.
\end{description}
\item[{Femur}] [thighbone] Longest and strongest bone in the body. 

The rounded upper end of each \emph{femur} fits securely into a socket in
a \emph{coxal bone}. The lower end of the femur intersects at the knee
joint with the larger of the two bones of the lower leg, the \emph{tibia},
which in turn makes contact with the thinner \emph{fibula}.
\item[{Tibia}] [shinbone] Bigger lower leg bone in front.
\item[{Fibula}] [calf bone] Smaller of the two lower leg bones in the back.

At the ankle, the \emph{tibia} and \emph{fibula} join with the seven \emph{tarsal
bones} that make up the ankle and heel. Five long bones, the
\emph{metatarsals}, form the foot. The 14 bones of the toes, like those of
the fingers, are called \emph{phalanges}.
\item[{Patella}] [kneecap] A triangle-shaped bone that protects and stabilizes
the knee joint.
\item[{Tarsals}] 

\item[{Metatarsals}] 

\item[{Phalanges}] 
\end{description}
\end{description}
\subsubsection{Joints form connections between bones}
\label{sec:orge808f43}
Structure and tissues that hold the skeleton together while still permitting
us to move about freely: \emph{joints}, \emph{ligaments}, and \emph{tendons}.

Joints are points of contact between bones. Fibrous joints are immovable in
adults, cartilaginous joints permit some movement, and synovial joints are
highly movable. Synovial joints are held together by ligaments and
lubricated by synovial fluid.

\begin{description}
\item[{Joints}] \{articulations\} The points of contact between bones.

Vary considerably from basically immovable to freely movable. Types of
joints include \emph{fibrous}, \emph{cartilaginous}, and \emph{synovial} joints.

\begin{description}
\item[{Fibrous joints}] Immovable.

\item[{Fontanels}] At birth, the flat bones in a baby's skull are separated by
relatively large spaces filled with fibrous connective tissue. These
"soft spots" are \emph{fontanels}.

The presence of joints also allows for brain growth and development
after birth. During childhood these fibrous joints gradually harden.
They are tiny, thin sutures by the time we are old.

\item[{Cartilaginous joints}] Bones are connected by \emph{hyaline cartilage}, they
are slightly movable and give some flexibility.

Examples include the cartilaginous joints that connect the
\emph{vertebrae} in the backbone, and those that attach the lower \emph{ribs}
to the \emph{sternum}.

\item[{Synovial joints}] The most freely moving joints are these. Bones are
separated by a thin fluid-filled cavity. The two bones are of a
synovial joint are fastened together and stabilized by \emph{ligaments}.

The interior is lined with a \emph{synovial membrane} which secretes
\emph{synovial fluid} to lubricate and cushion the joint. The
\emph{articulating} surfaces of the two bones are covered with a tough but
smooth layer of \emph{hyaline cartilage}. Together the synovial membrane
and the surrounding hyaline cartilage constitute the \emph{joint capsule}.

To reduce friction there are small disks of \emph{cartilage} on either
side of the knee called \emph{menisci}. The knee joint [hinge joint] also
includes 13 small sacs of fluid, called \emph{bursae}, for additional
cushioning.

\begin{description}
\item[{Synovial fluid}] 


\item[{Synovial membrane}] 


\item[{Joint capsule}] 


\item[{Hinge joint}] Like the knee or elbow, gets its name because it allows
movement in one plane like the hinges on a door.

\item[{Ball and socket joint}] Permits an even wider range of movement that
the \emph{hinge joint}.

Thanks to its design, a synovial joint can withstand tremendous
friction without wearing out. It gets its strength from
\emph{ligaments}, \emph{tendons}, and \emph{muscles}.
\end{description}
\end{description}

\item[{Ligaments and tendons}] Connective tissues that stabilize many joints.

Ligaments and tendons contain collagen arranged in parallel fibers,
making ligaments and tendons as strong and as flexible as a twisted
nylon rope. You can test how good these things are by loosening up your
legs and moving your \emph{patella} around. When you put tension in the
ligaments and tendons again though it doesn't move.
\end{description}

\subsubsection{Diseases and disorders of the skeletal system}
\label{sec:org584d907}
\subsection{Ch. 6}
\label{sec:orgbaeb300}
\subsubsection{Muscles produce movement or generate tension}
\label{sec:org15e2457}
40\% of body weight in males in muscle, 32\% in females.

\begin{description}
\item[{Skeletal muscles}] Most interact with the skeleton and cause bones to
move (or to prevent them from moving) relative to each other.
\begin{itemize}
\item Shivering, threading a needle, lifting heaving weights, standing
completely still, etc. are all tasks accomplish be skeletal muscles.
\item We have more that 600 skeletal muscles, often organized into pairs or
groups. Hundreds of muscles, each controlled by nerves and acting
either individually or in groups, produce all possible human motions.
Muscle groups that work together to create the same movement are
called \emph{synergistic muscles}. Muscles that oppose each other are called
\emph{antagonistic muscles}.
\end{itemize}
\item[{Smooth muscle}] [involuntary] Contracts/relaxes without your conscious
though.
\item[{Cardiac muscle}] [involuntary muscle] Makes up the heart.
\item[{Skeletal muscle}] [voluntary] Muscle tissue that is typically attached
to the skeletal muscle for movement.
\item[{Tendons}] Connects muscles to bones.
\item[{Ligaments}] Connects bones to bones.
\item[{Synergistic muscles}] Muscles that work together to create the same
movement.
\item[{Antagonistic muscles}] Muscles that oppose each other.
\item[{Origin}] End of muscle that attaches to relatively stationary bone.
\item[{Insertion}] End of muscle attached to another bone across a joint; the
"movable" end of the muscle.
\end{description}

\begin{enumerate}
\item Muscles
\label{sec:org2b0918c}
Muscles either produce or resist movement. Their fundamental activity is
\emph{contraction}. A muscle is composed of many muscle cells arranged in
parallel, each containing numerous \emph{myofibrils}. The contractile unit in a
myofibril is called a \emph{sarcomere}. A sarcomere contains think filaments of a
\emph{protein} called \emph{myosin} and thin filaments of a protein called \emph{actin}.

\begin{enumerate}
\item Internal
\label{sec:org9cbea4e}
\begin{description}
\item[{Myofibrils}] The nuclei of singular muscle cells need to be located just
under the cell membrane because nearly the entire interior of the cell
is packed with these long cylindrical structures arranged in parallel,
called \emph{myofibrils}.

Myofibrils are packed with contractile \emph{proteins} called \emph{actin} and
\emph{myosin}.

\item[{Sarcomere}] A segment of a \emph{myofibril} from one Z-line to the next is
called a \emph{sarcomere}.

A \emph{sarcomere} consists of two kinds of \emph{protein filaments}. Thick
filaments composed of a protein called \emph{myosin} are interspersed at
regular intervals within filaments of a different protein called
\emph{actin}.

\begin{description}
\item[{Actin}] 

\item[{Myosin}] 
\end{description}
\end{description}

\item Specific muscles
\label{sec:org65975dc}
\begin{description}
\item[{Buccinator}] Thin quadrilateral muscle, occupying the interval between
the maxilla and the mandible at the side of the face.

\item[{Orbicularis oris}] A complex of muscles in the lips that encircles the
mouth.

\item[{Zygomaticus major}] 

\item[{Frontalis}] 

\item[{Orbicularis oculi}] 

\item[{Masseter}] 


\item[{Sternocleidomastoid}] 

\item[{Pectoralis major}] Draws arm forward and toward the body
\item[{Deltoid}] Raises arm
\item[{Latissimus dorsi}] Rotates and draws arm backward and toward body
\item[{Trapezius}] Lifts shoulder blade, braces shoulder, draws head back.
\item[{Serratus anterior}] Helps raise arm, contributes to pushes, draws
shoulder blade forward

\item[{External intercostals}] 

\item[{Rectus abdominis}] Compress abdomen, bends backbone, compresses chest
cavity
\item[{External oblique}] Lateral rotation of trunk, compresses abdomen.
\item[{Transversus abdominis}] 

\item[{Biceps brachii}] Straightens forearm at elbow
\item[{Triceps brachii}] Lifts shoulder blade, braces shoulder, draws head back.
Rotates and draws arm backward and toward body.

\item[{Gluteus maximus}] Butt muscles; extends thigh, rotates thigh laterally.
\item[{Hamstrintgs}] helps your knee bend; draws thigh backward, bends knee
\begin{description}
\item[{Biceps femoris}] [not important]
\item[{Semimembranosus}] [not important]
\item[{Semitendinosus}] [not important]
\end{description}
\item[{Adductor Longus}] Flexes thigh, rotates thigh laterally, draws thigh
toward body. Inner thigh muscle.
\item[{Gastrocnemius}] Bends lower leg at knee, bends foot away from knee. Calf
muscles

\item[{Quadriceps femorus}] Flexes thigh at hips, extends leg at knee. Front of
upper leg area.
\begin{description}
\item[{Rectus femoris}] 

\item[{Vastus lateralis}] 

\item[{Vastus medialis}] 

\item[{Vastus intermedius}] 
\end{description}
\item[{Sartorius}] Rotates thigh outward.
\item[{Tibialis anterior}] Front of lower leg. Shin splints area
\item[{Achilles tendon}] Connects gastrocnemius muscle to heel.
\end{description}

\item Always contract, never push
\label{sec:org63f90cc}
\begin{itemize}
\item The can only Pull, never PUSH
\item Muscles contract in response to electrical or chemical stimuli.
\end{itemize}
\end{enumerate}

\item another word to eat
\label{sec:org2087c71}
\begin{description}
\item[{mastication}] another word for eating.

A lab T.A. said this would be on the test!
\end{description}

\item Movements
\label{sec:orgd53ad1d}
\begin{description}
\item[{Flexion}] Decreases angle of the joint between two structures.

\item[{Extension}] Increases angle of the joint between two structures.

\item[{Abduction}] Moves bone away from the midline of the body.

\item[{Adduction}] Moves bone toward the midline of body ("add" to the body's
midline).

\item[{Rotation}] Bone spins around its own axis.

\item[{Circumduction}] Bone describes a cone in three-dimensional space.

\item[{Supination}] Rotation of the forearm so palm faces anteriourly (imagine
holding a bowl of soup).

\item[{Pronation}] Rotation of the forearm so palm faces posteriorly.
\end{description}
\end{enumerate}

\subsubsection{Individual muscle cells contract and relax}
\label{sec:orgbdd690a}
During a muscle contraction each sarcomere shortens just a little. Subtle
though this action seems, it is also powerful. The contraction of an entire
skeletal muscle depends on the simultaneous shortening of the tiny
sarcomeres in its cells.

\begin{itemize}
\item A skeletal muscle cell must be activated by a nerve. It does not contract
on its own.
\item Nerve activation increases the concentration of calcium (Ca) in the
vicinity of the contractile proteins.
\item The presence of calcium permits contraction. The absence of calcium
prevents contraction.
\item When a muscle cell is no longer stimulated by a nerve, contraction ends.
\end{itemize}

\begin{enumerate}
\item Nerves activate skeletal muscles
\label{sec:orgc6d1416}
\begin{description}
\item[{Motor neurons}] Skeletal muscle cells are stimulated to contract by
these nerve cells.

The motor neuron secretes \emph{acetylcholine} (ACh) a \emph{neurotransmitter}.
The nerve cell has only either an excitatory or inhibitory effect.

\item[{Neuromuscular junction}] The junction between a motor neuron and a
skeletal muscle cell.

\emph{Acetylcholine} is released when there are electrical impulses
traveling in a motor neuron. Acetylcholine binds to receptor sites on
the muscle cell membrane, causing muscle cell membrane to generate an
electrical impulse traveling in all directions. \emph{T tubules} or
\emph{transverse tubules} transmit these random electrical impulses to all
parts of the cell as quickly as possible.
\end{description}

\item Activation releases calcium
\label{sec:org5d2a84a}
\begin{description}
\item[{Sarcoplasmic reticulum}] A series of membrane-bound chambers in close
contact to transverse tubules. It is forced to fit into the small
amount of space in the cell not occupied by myofibrils. The primary
function of the sarcoplasmic reticulum is to store ionic calcium
(Ca\(^{\text{2+}}\))

Latin \emph{sarco-} flesh or muscle
\end{description}

\item Calcium initiates the sliding filament mechanism
\label{sec:org93dee8e}
Contraction is inhibited unless calcium is present. Closely associated to
the \emph{actin} and \emph{myosin} are two other \emph{protein} molecules called
\emph{troponin} and \emph{tropomyosin}, together \emph{troponin-tropomyosin protein
complex}. 

\begin{description}
\item[{Sliding filament mechanism}] Muscles contract when sarcomeres shorten,
and sarcomeres shorten when the thick and thin filaments slide past
each other.
\end{description}
\end{enumerate}

\subsection{Ch. 9}
\label{sec:org671c426}
\textbf{The Immune System and Mechanisms of Defense}

\begin{description}
\item[{Pathogen}] Living organisms (bacteria) which cause disease.

Note that not \emph{all} bacteria are pathogens. Indeed, some are highly
beneficial.
\end{description}

\subsubsection{Self vs. non-self}
\label{sec:org7089e1d}
\begin{description}
\item[{B cell}] Each B cell produces unique B cell receptors which react only to
specific antigens. (Also applicable to T cells.) Created in bone marrow
with a random.
\end{description}

\subsubsection{Pathogens cause disease}
\label{sec:org31dc097}

\subsubsection{The lymphatic system defends the body}
\label{sec:orga4acf92}

\subsubsection{Keeping pathogens out: The first line of defense}
\label{sec:org58d9fef}
Various mechanisms create an inhospitable environment for pathogenic
microorganisms. \uline{Skin} is a dry outer barrier. \uline{Tears}, \uline{saliva}, \uline{earwax}, and \uline{mucus}
trap pathogens or wash them away. \uline{Acidic conditions} kill them or inhibit their
growth; \uline{urination}, \uline{defecation}, and \uline{vomiting} forcibly expel them; and \uline{resident
bacteria} compete with pathogens for food.

Most successful \emph{pathogens} \uline{enter} the body at places \uline{where} we do \uline{not} have \emph{skin}.
Enter through the mucous membranes that line the \emph{digestive}, \emph{urinary},
\emph{respiratory}, and \emph{reproductive} tracts; taking advantage of moist surfaces in
direct contact with living cells.

\begin{itemize}
\item Skin
\begin{enumerate}
\item \textbf{structure} has \emph{keratin} which is good
\item \textbf{constant replacement} sheds away pathogens on skin
\item \textbf{acidic pH} Low pH makes skin a hostile environment for many microorganisms.
\item production of \textbf{antibiotic} by sweat glands
\end{enumerate}
\item \textbf{tears}, \textbf{saliva}, and \textbf{earwax} contain \emph{lysozyme}. Saliva lubricates, and prevents
cracking.
\item \textbf{Mucus} Thick, gel-like material secreted by cells at various surfaces of the
body. Microorganisms coming in contact with mucus becomes mired and cannot
gain access to the cells beneath.
\item \textbf{Digestive} and \textbf{vaginal acids}. Undiluted, acids are strong enough to kill nearly
all pathogens.
\item \textbf{Vomiting}, \textbf{urination}, and \textbf{defecation}
\item \textbf{Resident bacteria} Help control the population of deadly pathogens by competing
with them successfully for food.
\end{itemize}


\begin{description}
\item[{Keratin}] Forms a dry, tough, somewhat elastic barrier to the entry of
microorganisms.
\item[{Lysozyme}] An enzyme that kills many bacteria.
\end{description}

\subsubsection{Nonspecific defenses: The second line of defense}
\label{sec:org6b2fd64}
\begin{center}
\includegraphics[width=.9\linewidth]{Ch. 9/immune_2016-11-16_23-18-44.png}
\end{center}
Nonspecific defense mechanisms involve a general attack against all foreign and
damaged cells. \emph{Neutrophils} and \emph{macrophages} engulf and digest bacteria and
damaged cells, and \emph{eosinophils} bombard larger organisms (too large to be
engulfed) with digestive enzymes. The inflammatory response attracts \emph{phagocytes}
and promotes \uline{tissue healing}. \emph{Interferons} interfere with viral reproduction, and
a modest \emph{fever} enhances our ability to fight infections.

If \emph{pathogens} manage to breach our physical and chemical barriers and start to
kill or damage cells, there is a problem. The body must actively \uline{seek out} the
\emph{pathogens} and get rid of them, then \uline{clean up} the injured area and \uline{repair} the
damage. \emph{Phagocytes} are \emph{white blood cells} that destroy foreign cells through the
process of \emph{phagocytosis}. A \emph{phagocyte} first captures a bacterium with its
cytoplasmic extensions, then draws the bacterium in, eventually engulfing it
(\emph{endocytosis}) and enclosing it in a membrane-bound vesicle. Powerful enzyme in
the \emph{lysosomes} dissolve the bacterial membranes and then the WBC jettisons the
bacterial wastes (\emph{exocytosis}).

\begin{description}
\item[{Monocyte}] (WBC) Largest type of leukocyte; can differentiate into
macrophages.
\item[{Phagocyte}] Cells which protect the body by ingesting (phagocytosing) harmful
foreign particles.
\item[{Phagocytosis}] Process in which white blood cells destroy foreign cells in
the body by engulfing it inside a membrane and digesting it with enzymes.
\item[{Lyososome}] Membrane-bound organelle found in animal cells. Spherical
vesicles containing enzymes which break down virtually all kinds of
biomolecules.
\item[{Neutrophils}] First white blood cells to respond to infection. Most abundant
type of white blood cell.
\item[{Macrophages}] White blood cells leaving the vascular system and entering into
tissue fluids. They then engulf and digest large numbers of foreign cells,
especially viruses and bacterial parasites. Technically no longer blood
cells, as they're no longer in the blood.
\end{description}
\begin{center}
\includegraphics[width=.9\linewidth]{Ch. 9/X2604-M-02_2016-11-16_23-20-10.png}
\end{center}
\begin{description}
\item[{Eosinophils}] WBC which take on invaders too big for \emph{phagocytosis}. They
cluster around large parasites such as flukes and pinworms and bombard them
with digestive enzymes. Also digest certain foreign \emph{protein}.
\end{description}
\begin{center}
\includegraphics[width=.9\linewidth]{Ch. 9/300px-Eosinophils_in_peripheral_blood_2016-11-16_23-16-38.jpg}
\end{center}
\begin{description}
\item[{Inflammation}] Four outward signs: redness, warmth, swelling, and pain.
Starts when tissue is injured. Triggers vasodilation which increases
capillary permeability for phagocytes to squeeze through.
\item[{Interferons}] Cells infected by viruses which secrete a group of proteins
called \emph{interferons} which diffuse to \uline{nearby healthy cells}, bind to their
cell membranes, and stimulate the healthy cells to produce proteins that
\uline{interfere with} the \uline{synthesis} of \emph{viral proteins}, making it harder for the
viruses to infect the protected cells.
\item[{Complement system}] (?) Comprises at least 20 plasma proteins that circulate
in the blood and complement, or assist, other defense mechanisms.
\item[{Fevers}] 
\end{description}

\subsubsection{Specific defense mechanisms: The third line of defense}
\label{sec:org2b30fbd}
\begin{center}
\includegraphics[width=.9\linewidth]{Ch. 9/800px-Hematopoiesis_simple.svg_2016-11-16_23-24-26.png}
\end{center}
An \emph{antigen} is any substance that provokes an \emph{immune response}. When activated by
first exposure to a specific \emph{antigen}, \emph{lymphocytes} called \emph{B cells} quickly produce
\emph{antibodies} against the \emph{antigen}. They also produce a few long-lived \emph{memory cells}
that remain inactive until the next exposure to the same \emph{antigen}. Other
\emph{lymphocytes} called \emph{T cells} mature in the \emph{thymus gland}. Helper \emph{T cells} stimulate
other immune cells, cytotoxic \emph{T cells} attack abnormal and foreign cells, and
memory \emph{T cells} store information until the next exposure to the same \emph{antigen}.

\begin{center}
\includegraphics[width=.9\linewidth]{Ch. 9/nonspecific-immune-response-3-638_2016-11-16_21-26-16.jpeg}
\end{center} \emph{First
exposure} to a specific antigen generates a \emph{primary immune response}. \emph{Subsequent
exposure} to the same antigen elicits a \emph{secondary immune response} that is faster,
longer lasting, and more effective than the primary immune response.

At first exposure to a particular antigen, the immune system generates a \emph{primary
immune response}. This involves recognition of the antigen, and production and
proliferation of \emph{B} and \emph{T cells}.

\begin{description}
\item[{Antigen}] A substance or part of a substance (living or non-living) that
the \emph{immune system} recognizes as \uline{foreign}. It \uline{activates} the \emph{immune system}
and \uline{reacts} with immune cells or their products, such as \emph{antibodies}. 

They are like keys for the lock, or antigen-binding site.
\item[{Antibody}] A \emph{protein molecule} released by a \emph{B cell} or plasma cell that \uline{binds
to} a specific \emph{antigen}.
\item[{Natural killer (NK) cells}] WBC (\emph{lymphocytes}) which target destroy tumor
cells and cells infected by viruses. NK cells are able to recognize certain
changes that take place in the plasma membranes of tumor cells and
virus-infected cells. "Natural killer" is more like "nonspecific killer."
\uline{Not} phagocytes, instead they release chemicals that break down their
targets' cell membranes.
\item[{Immunity}] B and T cells create a population of memory cells. This is the
\emph{secondary immune response} that is faster, longer lasting, and more
effective than the first.
\end{description}

\subsubsection{Immune memory creates immunity}
\label{sec:orga746543}
\subsubsection{Medical assistance in the war against pathogens}
\label{sec:org0ef081c}

\begin{description}
\item[{Active immunization}] Induction of immunity after exposure to a an \emph{antigen}.
Can occur naturally or can be artificial.
\item[{Vaccine}] Biological preparation that provides active acquired immunity to
particular disease. An \emph{artificial active immunization}.
\end{description}

\subsubsection{Tissue Rejection: A medical challenge}
\label{sec:orgb2a5382}

\subsubsection{Inappropriate immune system activity causes problems}
\label{sec:org6b195f0}

\subsubsection{Immune deficiency: The special case of AIDS}
\label{sec:orgfc2524c}

\subsection{Ch. 10}
\label{sec:orgc20807d}
\textbf{The Respiratory System: Exchange of Gases}
\subsection{Ch. 13}
\label{sec:org8eaba64}
\textbf{The Endocrine System}

The \emph{endocrine system} triggers \uline{sexual maturation}, sexual desire, \uline{uterine
contractions} during childhood, and milk letdown. It is involved in response to
stress, digestion, cellular metabolism, and overall organ growth and
development. Also is wholly or partly responsible for the maintenance of
homeostasis of many of the most important variables in the body, including salt
and water balance, blood pressure, the production of red blood cells, and blood
calcium concentration.

\subsubsection{The endocrine system produces hormones}
\label{sec:org3feb5e7}
\emph{Hormones} are bloodborne units of information, just as \emph{nerve impulses} are units
of information carried in \emph{nerves}. The hormones help to maintain \emph{homeostasis} by
regulating various body functions through \emph{feedback loops}. \emph{Hormones} secreted by
glands of the \emph{endocrine system} act only on \emph{target cells} with appropriate
receptors. \emph{Hormones} reach their targets via the \emph{circulatory system}, making
\emph{endocrine system} control slower than \emph{nervous system} control. The two systems
frequently interact.

\textbf{Endocrine has certain characteristics setting it apart from the nervous system
as a communications system}
\begin{enumerate}
\item \emph{Hormones} of the \emph{endocrine system} reach nearly every living cell. A distinct
advantage over the nervous system. Hormones conveniently circulate in the
\emph{blood} which is everywhere (except CNS).
\item Each \emph{hormone} acts only on certain cells. Each hormone acts only on a certain
group of cells, called its \emph{target cells}.
\item Endocrine control tends to be slower than nervous system. A consequence of
using the cardiovascular system as the message delivery system. Endocrine is
better with longer-term controls, such as regulation of blood pressure,
production of red blood cells, and onset of puberty.
\item The \emph{endocrine} and \emph{nervous systems} can (and often do) interact with each
other. The timing of growth and sexual maturation, for example, involves a
complex sequence of changes in both neural and endocrine signals, and the
release of some hormones depends on input from \emph{sensory neurons}.
\end{enumerate}


\begin{description}
\item[{Hormones}] Circulating chemical messenger molecules secreted by the \emph{endocrine
system}'s collection of \emph{specialized cells}, \emph{tissues}, and \emph{glands}.
\item[{Endocrine glands}] Where the \emph{hormones} are secreted. Ductless organs that
secrete their products into interstitial fluid, lymph, and blood.
\begin{itemize}
\item \emph{endocrine} means "secreted internally"
\end{itemize}
\item[{Target Cells}] The specific cell in the body with the appropriate receptor
for the targeting hormone. As an analogy consider a car needing a specific
set of car keys; when the key is put into the ignition it causes the car to
start up.
\end{description}

\subsubsection{Hormones are classified as steroid or non-steroid}
\label{sec:orgf36905c}
\emph{Hormones} participate in \emph{negative feedback loops}. As messenger molecules, some
\emph{hormones} participate in internal \emph{homeostatic} control mechanisms and control
vital physiological processes (they maintain \emph{homeostasis}). \emph{Steroid hormones}
enter the \emph{target cell}, activate specific genes, and cause the production of new
\emph{proteins}. \emph{Non-steroid hormones} bind to a \emph{cell membrane receptor} that either
opens or closes ion channels or activates a \emph{second messenger} within the cell.

\textbf{note}: Most of the \emph{hormones} discussed in this chapter are non-steroid. The only
\emph{steroid hormones} are those produced by the cortex of the \emph{adrenal gland} (cortisol
and aldosterone) and the sex hormones produced by the \emph{testes} (testosterone) and
\emph{ovaries} (estrogen and progesterone).

\begin{center}
\includegraphics[width=.9\linewidth]{Ch. 13/slide_8_2016-11-07_11-46-51.jpg}
\end{center}

\textbf{A negative feedback loop involving a hormone.} In response to a change in the
controlled variable, the endocrine gland releases a hormone that acts on target
cells to return the controlled variable to its normal state. Any change in the
controlled variable sets in motion a response that reverses that change.

\begin{description}
\item[{Steroid hormones}] Structurally related to \emph{cholesterol} (a \emph{lipid}), and
therefore they are lipid-soluble. \uline{Diffuses} into the \emph{target cells}.
Hormone-receptor complex attaches to \emph{DNA}, activating specific genes, then
causing the formation of \emph{messenger RNA}.
\item[{Nonsteroid hormones}] Structurally related to \emph{proteins}; therefore, \emph{lipid}
insoluble and cannot cross the \emph{cell membrane}. \uline{Bind to receptors} on outer
surface of \emph{target cell membranes}. Some cause ion channels in the \emph{cell
membrane} to open or close, \uline{similar} to the action of a \emph{neurotransmitter}.
\item[{Second messenger}] Commonly, hormone-receptor binding converts an inactive
molecule within the cell into an active molecule. The activated molecule
generated with the cell is this \emph{second messenger}.

A common second messenger is cyclic AMP, produced from ATP. Which then
activates an enzyme already present within the cell, which activates
another, which in turn activates another, and so on.

ATP \(\rightarrow\) Cyclic AMP (\emph{second messenger}) \(\rightarrow\) Enzyme 1 activated
\(\rightarrow\) Enzyme 2 activated \(\rightarrow\) Enzyme 3 activated \(\rightarrow\) Final product alters cell
activity
\end{description}

\subsubsection{The hypothalamus and the pituitary gland}
\label{sec:org566cbaf}
\begin{itemize}
\item 13.3 goes into a lot of hormones we do not need to know at this time like
Pituitary Hormones
\item focus on FSH and LH in conjunction with testosterone and negative feedback
\end{itemize}

Each anterior \emph{pituitary hormone} is produced and secrete by a separate cell type,
and each hormone is regulated by a separate mechanism. \emph{FSH} and \emph{LH} stimulate the
reproductive organs. They are called \emph{gonadotropins} because they stimulate the
growth, development, and function of the reproductive organs in both males and
females. In females, \emph{FSH} induces \uline{egg development}, \emph{LH} promotes ovulation (egg
release), and both hormones stimulate the secretion of the ovarian hormone
\emph{estrogen}. In addition, \emph{LH} stimulates the secretion of the ovarian hormone
\emph{progesterone} after ovulation. In males, \emph{FSH} induces \uline{sperm development} and \emph{LH}
stimulates the production of the hormone \emph{testosterone} by the testes.


\textbf{FSH \{pituitary gland\} \(\rightarrow\) SPERM/EGGS} \\
\textbf{LH \{pituitary gland\} \(\rightarrow\) TESTOSTERONE/(OVULATION/PROGESTERONE)} \\

\begin{description}
\item[{Hypothalamus}] Small region of the brain that serves as a homeostatic control
center.
\item[{Pituitary gland}] (master gland) \{hypophysis\} \emph{Endocrine gland} the size of a
pea and weighing 0.5 grams; protrusion off the \uline{bottom} of the \emph{hypothalamus}
at the \uline{base} of the \emph{brain}. \emph{Hormones} \uline{secreted} from the pituitary gland help
control: \uline{growth}, \uline{blood pressure}, certain functions of the \uline{sex organs},
\uline{thyroid glands}, and \uline{metabolism}, as well as some aspects of \uline{pregnancy},
\uline{childbirth}, \uline{nursing}, \uline{water/salt} concentration at the \uline{kidneys}, \uline{temperature
regulation} and \uline{pain relief}.
\item[{Follicle-stimulating hormone}] (FSH) \{gonadotropin, glycoprotein polypeptide\}
Synthesized and secreted by the anterior \emph{pituitary gland}. Regulates the
\uline{development}, \uline{growth}, \uline{pubertal maturation}, and \uline{reproductive processes} of the
body.
\item[{Luteinizing hormone}] (LH) \{lutropin/lutrophin\} Synthesized and secreted by
the anterior \emph{pituitary gland}. In females triggers ovulation. In males
(where LH had also been called interstitial cell-stimulating hormone) it
stimulates Leydig cell production of \emph{testosterone}.
\end{description}

\subsubsection{The pancreas secretes glucagon, insulin, and somatostatin}
\label{sec:org42f12cd}
[focus on insulin]

\emph{Endocrine cells} of the \emph{pancreas} are located in small clusters scattered
throughout the pancreas called the \emph{islets of Langerhans}. Insulin lowers glucose
levels by \uline{facilitating \emph{glucose} uptake} and \uline{storage}.

\begin{enumerate}
\item Alpha cells secrete \emph{glucagon}.
\item Beta cells secrete \emph{insulin}, which \uline{lowers blood sugar}. After a meal, \emph{blood
glucose} levels \uline{rise} as \emph{sugars} are \uline{absorbed} from the digestive tract. The \uline{high}
glucose concentration \uline{stimulates} the \emph{beta cells} to \uline{secrete} \emph{insulin} into the
blood, where it does the opposite of glucagon. \emph{Insulin} promotes the \uline{uptake} of
\emph{glucose} by cells of the \emph{liver}, \emph{muscle}, and \emph{fat tissue}. It also promotes the
conversion of \emph{glucose} \uline{into} \emph{glycogen} in the liver, both \emph{glycogen} and \emph{proteins}
in muscle, and fats in \emph{adipose tissue}.
\item Delta cells secrete somatostatin.
\end{enumerate}


\begin{description}
\item[{Pancreas}] Both an \emph{endocrine gland} (secreting hormone into the blood) and an
\emph{exocrine gland} (secreting \emph{enzymes}, \emph{fluids}, and \emph{ions} into the \emph{digestive
tract} to aid in digestion).
\end{description}

\subsubsection{The adrenal glands comprise the cortex and medulla}
\label{sec:org7a894c6}
[skip]

\subsubsection{Thyroid and parathyroid glands}
\label{sec:orgfa8a125}
[skip]

\subsubsection{Testes and ovaries produce sex hormones}
\label{sec:org719945e}
\begin{itemize}
\item focus on testes and ovaries
\end{itemize}

The human \emph{gonads} are the \emph{testes} of males and \emph{ovaries} of females. Responsible for
the \uline{production} of \emph{sperm} and \emph{eggs}; both organs are also \emph{endocrine glands} in that
they \uline{produce} the steroid \emph{sex hormones}.

\textbf{During puberty}: In males, the \emph{anterior pituitary gland} begins to release
\emph{luteinizing hormone} (\emph{LH}) which stimulates the testes to resume \emph{testosterone}
production. In females, the \emph{anterior pituitary} starts to release \emph{luteinizing
hormone} (\emph{LH}) and \emph{follicle-stimulating hormone} (\emph{FSH}). These hormones stimulate
the ovaries to begin secreting \emph{estrogen} and \emph{progesterone}.

\begin{center}
\includegraphics[width=.9\linewidth]{book/screenshot_2016-11-10_22-49-30.png}
\end{center}

\begin{description}
\item[{Testes}] Located in the \emph{scrotum}, produce \emph{androgens}, the male \emph{sex hormones}.
\item[{Ovaries}] Located in the abdomen, produces the female  sex hormones know
collectively as \emph{estrogens} (\emph{17b-estradiol}, \emph{estrone}, and \emph{estriol}), and also
\emph{progesterone}.
\end{description}

\subsubsection{Other glands and organs also secrete hormones}
\label{sec:orgab7231b}
[skip]

\subsubsection{Other chemical messengers}
\label{sec:org724a53a}
[skip]

\subsubsection{Disorders of the endocrine system}
\label{sec:org1cb1052}
[skip]
\subsection{Reproductive}
\label{sec:org0d69138}
\begin{itemize}
\item Epididymis
\item Vas deferens
\item[{Seminal Vesicles}] Secrete fructose to provide energy source.
\item[{Prostate}] Secretes alkaline fluid to neutralize vaginal acid for sperm passage.
\item[{Bulbourethral}] Secrete mucus into urethra to wash away acidic urine and
provide lubrication.
\item[{Penis}] Houses the urethra, the common route for both urine and sperm; male
sex organs.
\end{itemize}

\subsection{Ch. 14}
\label{sec:org14171ba}
\begin{itemize}
\item 14.1--14.7; particularly 14.1 and table 14.1
\item Read 14.8
\item How are macromolecules broken down, absorbed, and used?
\end{itemize}

\emph{The digestive system} and its accessory organs \uline{digest} and \uline{absorb} nearly
everything we eat and drink, regardless of how much we eat or drink. The stomach
\uline{stores ingested food} and \uline{water} until it can be \uline{delivered} to the \emph{small intestine};
also \uline{secreting} a \uline{strong acid} \emph{hydrochloric acid} (\emph{HCL}) that breaks down proteins
and most bacteria. Leftover non-absorbed waste products and bacteria are stored
until they are eliminated as feces. Nutrients and water are absorbed in the
small intestine (and lesserly to the large intestine). \emph{Enzymes} from the \emph{pancreas}
and the \emph{small intestine} \uline{break down} \emph{carbohydrates} and \emph{fats} to be absorbed. To
maintain a constant body weight, energy intake must equal energy expenditure.

In many ways the \emph{digestive system} is a highly efficient \uline{disassembly line}, taking
in food and processing it, breaking it into small pieces and digesting the
fragments with \emph{enzymes} and \emph{strong chemicals}.

\begin{center}
\includegraphics[width=.9\linewidth]{book/image777_2016-11-15_19-45-50.jpg}
\end{center}

\begin{center}
\includegraphics[width=.9\linewidth]{book/digestive-system-aldy-organ-donation-project_2016-11-15_19-44-36.jpg}
\end{center}

\begin{description}
\item[{nutrients}] Substances in food that are required for growth, reproduction,
and the maintenance of health.
\item[{Salivary amylase}] Breaks down carbohydrates (generally polysaccharides which
are a monosaccharide and dissacharide units linked by glycosidic bonds.)
\end{description}

\subsubsection{The digestive system brings nutrients into the body}
\label{sec:orgf645f37}

The digestive system consist of organs and accessory organs that share the
function of bringing nutrients into the body. The wall of the GI tract consists
of four tissue layers: the \emph{mucosa}, the \emph{submucosa}, the \emph{muscularis}, and the
\emph{serosa}. The five basic processes of digestion are:
\begin{enumerate}
\item mechanical processing and movement (mechanical digestion)
\item secretion
\item digestion (chemical digestion)
\item absorption
\item elimination
\end{enumerate}


Accessory Organs
\begin{itemize}
\item Don't directly contact food
\item Liver, Gallbladder, Pancreas, Salivary Glands
\end{itemize}

From the \emph{esophagus} to the \emph{anus}, the walls of the \emph{GI tract} \uline{share} common
\uline{structural features} consisting of \uline{four different layers}. Some of the organs of
the GI tract are separated from each other by thick rings of circular smooth
muscle called \emph{sphincters}, which close off passageways between \emph{organs}.

\begin{description}
\item[{The digestive system}] All the organs that share the common function of
getting nutrients into the body. Including a series of hollow organs
extending from the \emph{mouth} to the \emph{anus}: the \emph{mouth}, \emph{pharynx}, \emph{esophagus},
\emph{stomach}, \emph{small intestine}, \emph{large intestine}, \emph{rectum}, and \emph{anus}.
\item[{Gastrointestinal tract}] (\emph{GI tract}) The organs of the \emph{digestive system}
creating a hollow tube through the body. The area through which food and
liquids travel---the space within the tube---is called the \emph{lumen}.
\begin{description}
\item[{Mucosa}] (mucosal layer) Innermost tissue layer
\item[{Submucosa}] Next to the \emph{mucosa} is a \uline{layer of connective tissue} containing
\emph{blood vessels}, \emph{lymph vessels}, and \emph{nerves}, called the \emph{submucosa}.
\item[{Muscularis}] Third layer of GI tract tissue responsible for motility or
\uline{movement}. Two r three sub-layers of \emph{smooth muscle}.
\item[{Serosa}] Outermost layer of the GI tract wall is a \uline{thin connective tissue
sheath} that \uline{surrounds and protects} the \uline{other three layers}.
\end{description}
\end{description}


Five basic processes which accomplish digestive system function:
\begin{enumerate}
\item \emph{Mechanical processing and movement}. Chewing breaks food into smaller pieces,
and two types of movement (motility) mix the contents of the \emph{lumen} and
propel it forward.
\item \emph{Secretion}. Fluid, digestive \emph{enzymes}, \uline{acid}, alkali, bile, and mucus are all
secreted into the GI tract at various places. In addition, several hormones
that regulate digestion are secreted into the bloodstream.
\item \emph{Digestion}. The contents of the \emph{lumen} are \uline{broken down} both \uline{mechanically} and
\uline{chemically} into \uline{smaller and smaller particles}, culminating in \emph{nutrient
molecules}.
\item \emph{Absorption}. \emph{Nutrient molecules} pass across the \emph{mucosal layer} of the \emph{GI tract}
and into the \emph{blood} or \emph{lymph}.
\item \emph{Elimination}. \uline{Undigested material} is \uline{eliminated} from the body \uline{via} the \emph{anus}.
\end{enumerate}


The smooth muscle of the GI tract produces two kinds of motility, called
\emph{peristalsis} and \emph{segmentation}. \emph{Peristaltic waves} of contraction ripple through
the organs of the \emph{GI tract}, \uline{mixing} the contents of the \emph{stomach} and \uline{pushing} the
contents of the \emph{esophagus} and \emph{intestines} forward. Peristalsis is mostly
prevalent in the esophagus, where it transports food rapidly to the stomach.

\begin{description}
\item[{Peristalsis}] Propels food forward. The lump of food (\emph{bolus}) is \uline{pushed down}
the \emph{esophagus} because of \emph{smooth muscle} \uline{contracting before} the \emph{bolus} and
\uline{relaxing after} the \emph{bolus} around the portion of \emph{GI tract} near the \emph{bolus}.
\item[{Segmentation}] Mixes food. Short sections of \emph{smooth muscle} \uline{contract} and \uline{relax}
in \uline{seemingly random} fashion. The result is a \uline{back-and-forth mixing} of the
contents of the \emph{lumen}.
\end{description}

\begin{enumerate}
\item Other
\label{sec:orge5dd3f2}
\begin{description}
\item[{Small intestine}] \uline{Digests} \emph{proteins}, \emph{fats}, and \emph{carbohydrates}. \uline{Absorbs} most of
the \emph{water} and \emph{nutrients}. \uline{Secretes} digestive \emph{hormones} and \emph{enzymes}.
\item[{Sigmoid colon}] Stores feces.
\item[{Rectum}] Passageway for feces.
\item[{Anus}] Expels undigested material.
\end{description}
\end{enumerate}

\subsubsection{The mouth processes food for swallowing}
\label{sec:orgcecf9b6}
The four kinds of teeth (\emph{molars}, \emph{premolars}, \emph{canines}, and \emph{incisors}) \uline{mechanically
digest} chunks of food. \emph{Salivary glands} \uline{secrete} \emph{saliva}, which \uline{moistens food},
begins the \uline{chemical digestion} of \emph{carbohydrates}, \uline{maintains} the \emph{pH} of the \emph{mouth},
and \uline{protects} the teeth \uline{against} \emph{bacteria}.

The mouth is the entrance to the \emph{GI tract}. \emph{Digestion} begins in the \emph{mouth} with
the \uline{process of chewing}, which breaks food into \uline{smaller and smaller} chunks. The
\emph{mouth} functions as an effective \uline{food processor}. Chewing would be inefficient
without the muscular tongue, which positions food over the teeth and mashes it
against the roof of the mouth.

\begin{description}
\item[{Mouth}] Teeth chew food. \emph{Tongue} \uline{positions} and \uline{tastes} foods.
\begin{enumerate}
\item \textbf{Molars (12)} flat surfaces well adapted to grinding and crushing
\item \textbf{Premolars (8)}
\item \textbf{Canines (4)} tear food
\item \textbf{Incisors (8)} sharp-edged cut food
\end{enumerate}
\item[{Salivary glands}] (three pairs) Produce a watery fluid called \emph{saliva}. Saliva
moistens food, making it easier to chew and swallow.
\end{description}

\subsubsection{The pharynx and esophagus deliver food to the stomach}
\label{sec:org87862d5}
\emph{Swallowing} begins with \uline{voluntary movements} of the \emph{tongue}; the presence of food
\uline{initiates} an \uline{involuntary swallowing reflex}. \emph{Peristalsis} and gravity \uline{transfer}
food through the \emph{esophagus} \uline{to} the \emph{stomach}.

\uline{Voluntary movements} of the \emph{tongue} and \emph{jaws} \uline{push} a \emph{bolus} of food into the
\emph{pharynx}, stimulating receptors that initiate the "swallowing reflex," an
involuntary act that cannot be stopped once it is started. This \uline{closes off} the
passageways for air by closing off the passageway into the \emph{nasal cavity} and
bending the \emph{epiglottis} down to close the airway to the \emph{trachea}, temporarily
halting breathing.

\begin{description}
\item[{Pharynx}] (throat) Tongue pushes food down this to swallow.
\item[{Esophagus}] Just beyond the \emph{pharynx} is the \emph{esophagus}, a \uline{muscular tube}
consisting of both \emph{skeletal} and \emph{smooth muscle} that \uline{connects} the \emph{pharynx}
to the \emph{stomach}.
\end{description}

\subsubsection{The stomach stores food, digests protein, and regulates delivery}
\label{sec:orgf0915f4}
The \emph{stomach} \uline{stores} food, \uline{digests} it, and \uline{regulates} its delivery to the \emph{small
intestine}. \emph{Gastric juice} \uline{dissolves} \emph{connective tissue}, large \emph{proteins}, and
\uline{peptides} in food. The presence of \uline{food stretches} the \emph{stomach} and \uline{increases}
\emph{peristalsis}. \emph{Peristaltic contractions} \uline{mix} the \emph{chyme} and \uline{push} it gradually \uline{into}
the \emph{small intestine}.

The stomach is a muscular, expandable sac that performs the following three
important functions
\begin{enumerate}
\item \textbf{Food storage}. The \emph{stomach} \uline{stores} food until it can be digested and absorbed.
The \emph{stomach} \uline{shrinks when empty}, and then \uline{expands} to 1--3 liters of capacity
\uline{when eating}.
\item \textbf{Digestion}. The \emph{stomach} \uline{digests} \emph{proteins}, using strong \emph{acid}, and
protein-digesting \emph{enzymes}. The strong \emph{acid} also \uline{kills most bacteria}. \uline{Muscle
contractions mix} the secretions (\emph{acids}, \emph{enzymes}) with food, assisting in
mechanically \uline{breaking apart food particles}, and \uline{push the mixture} into the
\emph{small intestine}.
\item \textbf{Regulation of delivery}. The \emph{stomach} \uline{regulates} the \uline{rate} at which food is
\uline{delivered} to the \emph{small intestine}.
\end{enumerate}


\textbf{Gastric Juice breaks down the proteins.} Some of the cells lining the glands in
the lining of the stomach secrete either \emph{hydrochloric acid} (\emph{HCL}) or \emph{mucus}, but
most secrete \emph{pepsinogen}. Typically, the \emph{stomach} produces 1--2 liters of \emph{gastric
juice} per day, most of it immediately after meals, making the \emph{stomach pH}
approximately 2.

Some of the cells lining the \emph{stomach} and \emph{gastric glands} continuously \uline{produce} a
\uline{protective barrier} of \emph{mucus}, \uline{preventing} the \emph{stomach} from \uline{digesting itself}.

\begin{description}
\item[{Stomach}] Muscular, expandable sac.
\item[{Pepsinogen}] \emph{Pepsin's} proenzyme released by the chief cells in the stomach
wall which activates in the presence of \emph{hydrochloric acid} of the \emph{gastric
juice}.
\item[{Pepsin}] Becomes a protein-digesting \emph{enzyme} from \emph{pepsinogen} which is secreted
by the cells lining the \emph{gastric glands} after being exposed to \emph{acid} in the
\emph{stomach}.
\item[{Chyme}] Watery mixture of \uline{partially digested food} and \uline{gastric juice} delivered
to the \emph{small intestine}. The \emph{pyloric sphincter} between the \emph{stomach} and the
\emph{small intestine} regulates the rate of transport of \emph{chyme} into the \emph{small
intestine}.
\begin{description}
\item[{Cardiac sphincter}] Prevents acid reflux into \emph{esophagus}
\item[{Pyloric shpincter}] Regulates passage of chyme to \emph{small intestine}.
\end{description}
\item[{Peptic ulcer}] Open, (sometimes) bleeding \uline{sore} which occasionally occur in
the \emph{esophagus} and upper part of the \emph{small intestine} as well.
\end{description}


\textbf{Stomach contractions mix food and push it forward}. While \uline{empty}, the \emph{stomach's}
\uline{muscle contractions} keep it \uline{small}. When eating, the \uline{muscle contractions cease}
and \emph{stomach} \uline{relaxes}, \uline{signaling} \emph{peristalsis} to \uline{increase}. \emph{Chyme} with a \uline{high acid}
or \uline{fat} content \uline{stimulates} the \uline{release} of \emph{hormones} that \uline{slow} \emph{stomach peristalsis}.

\subsubsection{The small intestine digests food and absorbs nutrients and water}
\label{sec:orgae79584}
The \emph{small intestine} has two major functions: 1) \uline{digesting} \emph{proteins},
\emph{carbohydrates}, and \emph{lipids}, and 2) \uline{absorbing} approximately \uline{90\%} of the \emph{nutrients}
and \emph{water} we consume. Projections called \emph{villi} in the \emph{mucosa} \uline{increase} the \emph{small
intestine's} \uline{surface area} for absorption.

The process of \uline{digestion continues} in the \emph{small intestine}. It has two major
functions:
\begin{enumerate}
\item \emph{Digestion}. The \emph{stomach} partially \uline{digests} \emph{proteins} to smaller \emph{peptides}, under
the \uline{influence of} strong \emph{acids} and \emph{pepsin}. \emph{Protein} \uline{digestion continues} in the
\emph{small intestine}, but also \uline{digests} \emph{carbohydrates} and \emph{lipids}. Digestion of
\emph{protein}, \emph{carbohydrates}, and \emph{lipids} in the \emph{small intestine} involves
\uline{neutralizing} the highly acidic \emph{gastric juice} and \uline{adding} additional digestive
\emph{enzymes} from the \emph{intestine} and \emph{pancreas}.
\item \emph{Absorption}. Eventually, the \emph{proteins}, \emph{carbon}, and \emph{lipids} in food are \uline{broken
down} to single \emph{amino acids}, \emph{monosaccharides}, \emph{fatty acids}, and \emph{glycerol}, which
are small enough to be \uline{transported} (i.e. NOT DIFFUSED) across mucosal cells
into the blood. Nearly 90\% of the \emph{nutrients} able to be absorbed and \emph{water} is
\uline{absorbed in} the \emph{small intestine}.
\end{enumerate}


The \emph{small intestine} consists of three different regions. The \emph{duodenum}, the
\emph{jejunum}, the \emph{ileum}.

\begin{description}
\item[{Duodenum}] About \uline{10 inches long}. Most of the \emph{digestion} \uline{takes place} here. The
\emph{products of digestion} are \uline{absorbed} primarily in the \uline{other two segments}, the
\emph{jejunum} and the \emph{ileum}, which together are about \uline{10 feet long}.
\item[{Peptide}] Basically just small \emph{proteins}.
\item[{Villus}] (plural: \textbf{villi}) The mucosa contains large folds covered with
\uline{microscopic projections} called \emph{villi}, which each \emph{epithelial cell} of the
\emph{villi} has \uline{dozens} of even \uline{smaller}, \emph{cytoplasmic projections} called
\emph{microvilli}. The folds increase the surface area for absorption.
\end{description}

\subsubsection{Accessory organs aid digestion and absorption}
\label{sec:org3bde650}
The \emph{pancreas} \uline{secretes} \emph{digestive enzymes} and \emph{sodium bicarbonate}. The \emph{sodium
bicarbonate} \uline{neutralizes} \emph{stomach acid}, making the \emph{digestive enzymes} \uline{more
effective}. The \emph{liver} \uline{produces} \emph{bile}, which is \uline{stored} in the \emph{gallbladder} until
after a meal. The \emph{liver} also \uline{produces} \emph{plasma proteins}; \uline{inactivates} \emph{toxic
chemicals}; \uline{destroys} old \emph{red blood cells}; \uline{stores} \emph{vitamins}, \emph{iron}, and certain
products of \emph{metabolism}; and performs \uline{other functions} important for \emph{homeostasis}.

The \emph{digestive system} has four \emph{accessory organs}: \emph{salivary glands}, \emph{pancreas},
\emph{gallbladder}, and \emph{liver}.

\textbf{The pancreas secretes enzymes and NaHCO\(_{\text{3}}\)}. The exocrine products of the
pancreas are:
\begin{enumerate}
\item \emph{Sodium bicarbonate}. Except for \emph{pepsin}, most \emph{digestion enzymes} work \uline{best} at a
fairly \uline{neutral \emph{pH}}. \emph{sodium bicarbonate} from the \emph{pancreas} neutralizes the
\emph{stomach acid} in the \emph{small intestine} so that further digestion can proceed.
\item \emph{Digestive enzymes}
\begin{description}
\item[{\emph{proteases}}] Enzymes that digest proteins.
\item[{\emph{pancreatic amylase}}] Continues the digestion of carbohydrates only
partially accomlished by salivary amylase.
\item[{\emph{lipase}}] \emph{Lipid-digesting enzyme}. The \emph{small intestine} does not have lipase,
so the this \uline{\emph{pancreas} product} is important.
\end{description}
\end{enumerate}

The \emph{liver} serves a number of functions that \uline{maintain} \emph{homeostasis}:
\begin{itemize}
\item Storing fat-soluble vitamins (A, D, E, and K) and iron.
\item Storing glucose as glycogen after a meal, and converting glycogen to glucose
between meals.
\item Manufacturing plasma proteins, such as \emph{albumin} and \emph{fibrinogen}, from \emph{amino
acids}.
\item Synthesizing and storing some lipids.
\item Inactivating many chemicals, including \emph{alcohol}, \emph{hormones}, \emph{drugs}, and \emph{poisons}.
\item Converting \emph{ammonia} (\emph{NH\(_{\text{3}}\)}), a toxic waste product of metabolism, into less
toxic \emph{urea}.
\item Destroying worn-out red blood cells
\end{itemize}

\emph{Liver} injury can be particularly dangerous. Overexposure to \emph{toxic chemicals},
\emph{medications}, or \emph{alcohol} \uline{can damage} the \emph{liver} because it takes up these
substances to "detoxify" them, \uline{killing} some \emph{liver cells} in the process.
Long-term exposure \uline{can destroy} enough \emph{cells} to \uline{permanently impair} \emph{liver}
function, a condition known as \emph{cirrhosis}.


\begin{description}
\item[{Pancreas}] Elongated organ which lies just behind the stomach. Secretes
digestive \emph{enzymes} (\emph{proteases}, \emph{amylase}, \emph{lipase}) into \emph{small intestine}; most
of the enzymes involved in digestion come from the pancreas (the small
intestine contributes only a limited amount of protein- and
carbohydrate-digesting enzymes). Secretes \emph{sodium bicarbonate} into \emph{small
intestine} to \uline{neutralize} \emph{stomach acid}. \uline{Delivers} \emph{enzymes} and \emph{sodium
bicarbonte} to the \emph{duodenum} \uline{via} \emph{ducts}. The most important hormones it
secretes into the blood are \emph{insulin} and \emph{glucagon}.

The \emph{pancreas} is also an \emph{exocrine gland} that \uline{produces} and \uline{secretes} several
products directly into \emph{ducts} \uline{leading to} the \emph{digestive tract}.
\item[{Sodium Bicarbonate}] (Baking Soda) Natural substance in the body within the
bloodstream that \uline{regulates} \emph{pH} as a counterbalance to acid build up. Also a
first line of defense for a vast range of sickness like cancer, flu,
diabetes, kidney disease, and the common cold.
\item[{Liver}] Upper-right \emph{abdominal cavity}. Produces \emph{bile} (\emph{water} and \emph{electrolytes},
\emph{cholesterol}, \emph{bile salts}, \emph{lecithin}, and \emph{pigments}). Performs various
functions associated with \uline{processing} and \uline{storing} \emph{nutrients}.
\item[{Bile}] Watery mixture containing \emph{electrolytes}, \emph{cholesterol}, \emph{bile salts}
\uline{derived from} \emph{cholesterol}, a \emph{phospholipid} called \emph{lecithin}, and \emph{pigments}
(primarily \emph{bilirubin}) \uline{derived from} the breakdown of \emph{hemoglobin}.
\item[{Hepatic portal system}] Generally, a \emph{portal system} \uline{carries} \emph{blood} from one
\emph{capillary bed} to another. Here, the hepatic portal system carries
nutrient-rich blood directly from the digestive organs to the liver via the
hepatic portal vein.
\item[{Gallbladder}] \emph{Bile} produced by the \emph{liver} flows through \emph{ducts} to the
\emph{gallbladder}. \uline{Stores} and \uline{concentrates} \emph{bile}. Delivers bile to the \emph{duodenum}
via the common \emph{bile duct}. Concentrates \emph{bile} by \uline{removing} most of the \emph{water},
then \uline{stores} the \emph{concentrated bile} until it is needed.
\end{description}

\subsubsection{The large intestine  absorbs nutrients and eliminates wastes}
\label{sec:orgc92be6d}

\begin{description}
\item[{Large intestine}] 

\item[{Appendix}] 

\item[{Colon}] 
\end{description}

\subsubsection{How nutrients are absorbed}
\label{sec:org25e29c7}
\subsection{Ch. 15}
\label{sec:org2406e5d}
\begin{itemize}
\item Sections 15.1 to 15.4, 15.8 (page 356--364, 371--373 in 7\(^{\text{th}}\) ed.)
\item Figure 15.1 and figure 15.6
\item \textbf{Note}: You do NOT need to memorize blood vessels supplying the kidneys---focus
on the \uline{nephrons} and where/how \uline{filtration}, \uline{reabsorption} and \uline{secretion} take
place.
\item Overarching goal: Explain the role of the urinary system (\(_{\text{*}}\)especially
kidneys*\_) in maintaining homeostasis.
\end{itemize}

\begin{center}
\includegraphics[width=.9\linewidth]{Ch. 15/screenshot_2016-11-15_19-11-15.png}
\end{center}

\subsubsection{The urinary system contributes to homeostasis}
\label{sec:org003e9d6}
\textbf{Organ systems involved in removing wastes and \uline{maintaining homeostasis} of \emph{water}
and \emph{solutes}.} The \emph{kidneys} of the \emph{urinary system} are the organs primarily
responsible for the \uline{mainenance of homeostasis} of water and solutes and for the
excretion of most waste products.

\textbf{Table 15.1} 
\begin{center}
\begin{tabular}{ll|lr|}
\hline
Water gain (ml/day) &  & Water loss (ml/day) & \\
\hline
Drinking fluids & 1,000 & Urine & 1,500\\
Water in food & 1,200 & Evaporative loss (lungs) & 500\\
Metabolic Production & 300 & Evaporative loss (skin) & 350\\
 &  & Feces & 150\\
\hline
Total & 2,500 & Total & 2,500\\
\hline
\end{tabular}
\end{center}


The \emph{urinary system} maintains a constant internal environment by \uline{regulating} \emph{water
balance} and body levels of \emph{nitrogenous wastes}, \emph{ions}, and other substances. It
\uline{filters} metabolic wastes from the \emph{blood} and \emph{excretes} them in \emph{urine}. The major
\emph{nitrogenous waste} product is \emph{urea}.

\begin{itemize}
\item The digestive system + nutrients + water \(\rightarrow\) food residues
\item Lungs + O\(_{\text{2}}\) \(\rightarrow\) carbon dioxide gas
\item Skin + heat \(\rightarrow\) water + salt
\item liver + numerous inactivate-able substances \(\rightarrow\) (sometimes) bile in feces
\end{itemize}


\textbf{The kidneys regulate water levels}. The kidneys have a tremendous capacity to
adjust water excretion as necessary, from a minimum of about 1/2 liter per day
to nearly 1 liter per hour.

\textbf{The kidneys regulate nitrogenous wastes and other solutes}. The primary solutes
excreted by the \emph{kidneys} are \emph{nitrogenous wastes}, excess ions, and trace amounts
of other substances. The normal \uline{metabolism of proteins} leaves us with \uline{excess}
\emph{nitrogenous wastes}, which must be \emph{\uline{excreted}} by the \emph{kidneys}.

\begin{itemize}
\item It \uline{regulates} \emph{urinary excretion} of these \emph{ions}:
\begin{description}
\item[{NA\(^{\text{+}}\) and CL\(^{\text{-}}\)}] determine the volume of the extracellular fluids,
including blood.
\item[{K\(^{\text{+}}\)}] Maintains electrical charges
\item[{CA\(^{\text{2+}}\)}] Important in nerve and muscle activity
\item[{H\(^{\text{+}}\)}] Maintains acid--base balance.
\item[{Creatinine}] Produced during metabolism of creatine phosphate which gives
urine its yellow color
\end{description}
\end{itemize}


\begin{description}
\item[{Excretion}] Process that \uline{removes wastes} and excess materials from the body.
\item[{Urinary system}] Consists of the two \emph{kidneys}, the \emph{ureters}, the \emph{bladder}, and
the \emph{urethra}.
\item[{Kidney}] Two \emph{kidneys} \uline{produce urine}. carefully \uline{regulate} how much \uline{water} and
\uline{salt} (\emph{ions}) are \emph{excreted} in order to \uline{maintain homeostasis} of fluid volume
and blood pressure. Retain within the body all three classes of \emph{nutrients}
(\emph{carbohydrates}, \emph{lipids}, and \emph{proteins})
\begin{description}
\item[{Bowman's capsule}] Cup underneath the glomerulus which catches the
filtrate(is this the right word?).
\item[{Bowman's space}] Space inside Bowman's capsule.
\item[{Endothelial cells}] (Renestrated, or has a lot of holes/pores)
\item[{Basement membrane}] membrane underneath the endothelial cells.
\item[{Tubular cells}] (epithelial cell) Action point at the end of Bowman's
capsule.
\end{description}
\item[{Urine}] Waste water and waste solutes (various ions, drugs, vitamins, toxic
chemicals, virtually every small waste molecule produced anywhere in the
body).
\item[{Urea}] (H\(_{\text{2}}\)N--CO--NH\(_{\text{2}}\)). Created with water from ammonia molecules (which are
toxic to cells) and carbon dioxide.
\item[{Ammonia}] (NH\(_{\text{4}}^{\text{+}}\))
\item[{Diuretics}] (called water pills) Any substance which promotes an increased
production of urine. Treat a variety of conditions, such as high \emph{blood
pressure}, \emph{glaucoma}, and \emph{edema}. Help rid the body of salt (sodium) and
water. \emph{Kidneys} release more sodium into \emph{urine}.
\item[{Persistent proteinuria}] Urine has proteins in it which are not supposed to
be in urine. This might happen in somebody with \emph{kidney disease}---as it is
the job of the kidney not to put proteins in the urine.
\end{description}

\subsubsection{Organs of the urinary system}
\label{sec:org177181f}
Organs of the \emph{urinary system} include the \emph{kidneys}, \emph{ureters}, \emph{bladder}, and \emph{urethra}.
The \emph{kidneys} are the principal urinary organs, although they have several
\uline{homeostatic functions} as well. The \emph{ureters} transport \emph{urine} to the \emph{bladder}, where
it is stored until carried by the \emph{urethra} to the body's eternal opening.

\begin{center}
\begin{tabular}{l|ll}
Organ &  & Function\\
\hline
Kidneys &  & Excrete metabolic wastes, especially urea\\
 &  & Maintain water and salt homeostasis\\
 &  & Help regulate acid--base balance\\
 &  & Help regulate blood pressure (produce renin, an enzyme)\\
 &  & Control RBC production (produce erythropoietin, a hormone)\\
 &  & Activate vitamin D\\
\hline
Ureters &  & Transport urine to bladder\\
\hline
Urinary bladder &  & Stores urine until excretion\\
\hline
Urethra &  & Transports urine to outside the body\\
\end{tabular}
\end{center}

\textbf{Kidneys: the principal urinary organs}

The \emph{kidneys} are located on either side of the \uline{vertebral column}, near the
\uline{posterior body wall}. A \emph{renal artery} and a \emph{renal vein} connect each kidney to the
aorta and inferior vena cava, respectively.

\textbf{Ureters transport urine to the bladder}

\textbf{Urinary bladder stores urine}

Consists of three layers of \emph{smooth muscle} lined on the inside by \emph{epithelial
cells}. Typically holds about 600--1,000 ml of \emph{urine}.

\begin{description}
\item[{Medulla}] 

\item[{Cortex}] 

\item[{Renal pelvis}] 

\item[{Ureter}] Muscular tube that transports urine to the bladder. About 10--15
seconds a peristaltic wave of smooth muscle contraction occurs pushing
urine from the 10-inch length of the \emph{ureters} to the \emph{bladder}.
\item[{Urethra}] During urination, \emph{urine} passes through \emph{this}, a single, muscular
tube extending from the \emph{bladder} to the body's external opening.
\end{description}

\subsubsection{Nephrons produce urine}
\label{sec:org7d30e64}
\begin{itemize}
\item Yeah, so I really doubt that most of the big words in the this section are
actually going to be useful. Basically, there are nephrons in the kidney that
pull blood into the kidney where the nephron then takes out a tiny piece of
garbage and inserts it into the collecting duct with some water, calling it
urine.
\end{itemize}

A nephron is the function unit of a kidney. A nephron tubule consists of a
glomerular capsule, where fluid is filtered, and four reguions in which the
filtrate is modified before it becomes urine: proximal tubule, loop of Henle,
distal tubule, and collecting duct. Blood flows to the glomerulus via the renal
artery and afferent arteriols. Peritubuluar capillaries carry the blood to the
proximal and distal tubules, and vasa recta supply the loops of Henle and
collecting ducts.

Each kidney contains approximately a million small functional units called
\emph{nephrons}. An individual \emph{nephron} consists of a thin, hollow tube of \emph{epithelial
cells}, called a \emph{tubule}, plus the \emph{blood vessels} that supply the \emph{tubule}.

\begin{description}
\item[{Nephron}] 

\item[{Glomerular capsule}] 

\item[{Glomerulus}] 

\item[{Proximal tubule}] 

\item[{Loop of Henle}] 
\end{description}

\subsubsection{Formation of urine: Filtration, reabsorption, and secretion}
\label{sec:org1c11820}
\begin{itemize}
\item In the first part, proteins are separated from the solutes and water; solutes
and water placed into tubule. Most of what is extracted is then
reabsorbed---nearly all the filtered water and sodium and all the major
nutrients. In the third part, the waste products are placed into the tubule to
be secreted as urine.
\end{itemize}

\begin{center}
\includegraphics[width=.9\linewidth]{Ch. 15/figure_25_09_labeled_2016-11-16_20-43-29.jpg}
\end{center}

\textbf{Figure 15.6} The three processes that contribute to the formation of urine.
\begin{enumerate}
\item \textbf{Glomerular filtration} The movement of a \emph{protein}-free solution of fluid and
solutes from the \emph{glomerulus} into the space within the \emph{glomerular
capsule}. Water, ions, glucose, amino acids, bicarbonate, and waste products
(urea, creatinine) are \uline{filtered} from the \emph{glomerular capillaries} into the
space within the \emph{glomerular capsule}.
\item \textbf{Tubular reabsorption} The return of most of the fluid and solutes back into
the \emph{peritubular capillaries} or \emph{vasa recta}. Water, amino acids, glucose, most
ions (including bicarbonate), and some urea are \uline{reabsorbed} back into the
\emph{peritubular capillaries}, primarily in the \emph{proximal tubule} but also in other
\emph{nephron} segments.
\item \textbf{Tubular secretion} The addition of certain solutes from the \emph{peritubular
capillaries} or \emph{vasa recta} into the tubule. Some drugs, waste products, and
ions (primarily hydrogen, ammonium, and potassium) are \emph{actively secreted} from
the \emph{peritubular capillaries} primarily into the \emph{distal tubule} but also in
other \emph{nephron} segments.
\end{enumerate}


\begin{itemize}
\item \emph{Efferent} arterial is much \uline{narrower} than the \emph{afferent} arterial, and the
\emph{efferent} arterial has pretty \uline{high pressure}. The kidneys thus do \uline{not need} to
use any \uline{energy} in the \uline{\emph{filtration} process} because diffusion is so easy.
\end{itemize}

\textbf{The rate of filtration is regulated in two ways}:
\begin{itemize}
\item Under \uline{resting conditions}, pressure-sensitive cells in the \emph{arterioles} and
flow-sensitive cells in the \emph{tubule} walls can \uline{release chemicals} to \uline{adjust the
diameter} of the \emph{afferent arterioles}. These feedback mechanisms maintain a
relatively constant rate of \emph{glomerular filtration}, allowing the kidneys to
carry out their regulatory functions.
\item During times of stress (such as after an injury or while running a marathon),
\uline{blood flow} to the \emph{kidneys} \uline{falls} substantially as blood is redistributed to
more critical organs. The \emph{sympathetic} division of the \emph{autonomic nervous system}
constricts \emph{afferent} and \emph{efferent} \emph{arterioles}, reducing blood flow and the rates
of \emph{glomerular filtration} and urine formation. The \emph{kidneys} are unharmed because
they do not need a high blood flow to survive, and the body can cut back on
urine production temporarily without ill effects.
\end{itemize}
\subsubsection{Disorders of the urinary system}
\label{sec:orgbe9262b}
\subsection{Ch. 16}
\label{sec:org56de56c}
\begin{description}
\item[{Oxytocin}] [hormone, neuropeptide, medication] Normally produced in the
hypothalamus and released by the posterior pituitary. Plays a role in
social bonding, sexual reproduction in both sexes, and during and after
childbirth.
\end{description}

\subsubsection{The male reproductive system delivers sperm}
\label{sec:orgba2b3c5}
The male reproductive system comprises the \emph{testes}, the \emph{penis}, and associated
ducts and glands. \emph{Semen} consists of \emph{sperm} and three glandular secretions that
provide energy and the proper pH environment for the \emph{sperm} and also lubrication
for sexual intercourse. Millions of \emph{sperm} form every day throughout a man's
life; a typical ejaculate contains up to 300 million \emph{sperm}. \emph{Testosterone}
stimulates the growth and function of the male reproductive system, and
encourages aggressive and sexual behavior. Blood level of \emph{testosterone} are
regulated by a \emph{negative feedback loop} involving \emph{GnRH} from the \emph{hypothalamus} and
\emph{LH} and \emph{FSH} from the \emph{anterior pituitary}.

\begin{description}
\item[{Sperm}] 

\item[{Egg}] 

\item[{Testes}] 

\item[{Scrotum}] 
\end{description}


Each testis is only about 2 inches long, but it contains over 100 yards of
tightly packed \emph{seminiferous tubules}. Seminiferous tubules join to become the
\emph{epididymis}.

\begin{description}
\item[{Seminiferous tubules}] Site of sperm production.
\item[{Epididymis}] Tube that connects a \emph{testicle} to a \emph{vas deferens}.
\item[{Ductus (vas) deferens}] 

\item[{Ejaculatory duct}] 


\item[{Penis}] 

\item[{Erection}] 

\item[{Semen}] 
\end{description}

The entire process of sperm formation and maturation takes about 9--10 weeks.

\begin{description}
\item[{Seminal vesicles}] Produce seminal fluid, a watery mixture containing
fructose and prostaglandins that represents about 60\% of the volume of
semen.
\item[{Prostate gland}] 

\item[{Bulbourethral glands}] 

\item[{Gametes}] Cells having only 23 chromosomes. Female gametes are called
\emph{oocytes}.
\item[{Sertoli cells}] Large cells which nourish spermatogonia which are to begin
sperm.

\item[{Testosterone}] 

\item[{Gonadotropin-releasing hormone}] (GnRH)
\item[{Luteinizing hormone}] (LH)
\item[{Inhibin}] Highly active sertoli cells secrete a hormone called inhibin that
directly inhibits the secretion of FSH.
\end{description}

\subsubsection{The female reproductive system produces eggs and supports pregnancy}
\label{sec:org8a33cb8}
The \emph{ovaries} secrete \emph{estrogen} and \emph{progesterone}, store immature \emph{oocytes}, and
(usually) release one \emph{oocyte} at a time at intervals of about 28 days. The \emph{oocyte}
travels through the \emph{oviduct} to the \emph{uterus}, where implantation occurs if the egg
has been fertilized. The \emph{vagina} is the female organ of sexual intercourse and
the birth canal; around its opening are the structures of the \emph{vulva}. \emph{Mammary
glands} are accessory organs that produce and store milk.

\begin{description}
\item[{Oviduct}] The tube through which an \emph{ovum} or egg passes from an \emph{ovary}. In
female mammals this passageway is know as the \emph{uterine tube} or \emph{Fallopian
tube}.
\item[{Ovaries}] 

\item[{Estrogen}] 

\item[{Progesterone}] 

\item[{Oviduct}] 


\item[{Uterus}] 

\item[{Endometrium}] 

\item[{Cervix}] 

\item[{Vagina}] 

\item[{Labia majora}] Two prominent longitudinal cutaneous folds that extend
downward.
\item[{Labia minora}] (inner labia) Folds of skin in the very middle of \emph{vulva}
between thicker \emph{labia majora}.
\item[{Clitoris}] 


\item[{Mammary glands}] Accessory organs that produce and store milk.
\item[{Lactation}] 
\end{description}

\subsubsection{Menstrual cycle consists of ovarian and uterine cycles}
\label{sec:org3dd8f6d}
During the \emph{ovarian cycle}, a primary \emph{oocyte} within a developing \emph{follicle} divides
once to form a secondary \emph{oocyte}. The \emph{follicle} ruptures, releases the \emph{oocyte}, and
forms the \emph{corpus luteum} that secretes \emph{progesterone} and \emph{estrogen}. Rising levels
of \emph{estrogen} cause the \emph{endometrium} to proliferate. If pregnancy does not occur,
hormone levels fall and the \emph{endometrial layer} disintegrates and is shed, a
process known as \emph{menstruation}. \emph{Ovulation} is triggered by a surge of \emph{LH}, which in
turn is caused by the \emph{positive feedback} effect of a high concentration of
\emph{estrogen} from the maturing \emph{follicle}. During the second half of the \emph{menstrual
cycle}, sustained high levels of \emph{estrogen} and \emph{progesterone} from the \emph{corpus luteum}
inhibit further \emph{ovulation}.

\begin{description}
\item[{Menstrual cycle}] 

\item[{Ovarian cycle}] 

\item[{Follicle}] 

\item[{Ovulation}] 

\item[{Corpus luteum}] 

\item[{Menstruation}] 
\end{description}

\subsubsection{Human sexual response, intercourse, and fertilization}
\label{sec:org0a3c06d}
Women and men experience the same four phases of sexual responsiveness. Sexual
arousal in the male results in penile erection that leads to orgasm and
ejaculation. Females experience sexual arousal and pleasurable orgasms marked by
rhythmic muscular contractions. During ejaculation, the male deposits several
hundred million sperm in the vagina. Fertilization of the egg by a single sperm
occurs within five days, if it occurs at all.

\begin{description}
\item[{Orgasm}] 

\item[{Ejaculation}] 
\end{description}

\subsubsection{Birth control methods: Controlling fertility}
\label{sec:org2a01029}
Surgical sterilization should be considered a permanent method of birth control.
Hormonal methods---pills, injections, patches, and rings---are also relatively
effective but can have side effects. Physical barriers (diaphragms, cervical
caps, and condoms) and chemical spermicides are moderately effective; a few
afford some protection against diseases. IUDs are fairly effective against
pregnancy but do not protect against diseases. Withdrawal and periodic
abstinence are not effective forms of birth control in the long term. Abortion
is an elective but controversial procedure that terminates a pregnancy.

\subsubsection{Infertility: Inability to conceive}
\label{sec:org6fc4d55}
Male infertility is an insufficiency or lack of sperm. Causes of female
infertility are variable and include failure to ovulate, damage to oviducts,
pelvic inflammatory disease, secretions that impair sperm function, uterine
tumors, endometriosis, age-related changes, and miscarriages. The choice of
options to improve fertility depends on the cause of the infertility. Options
include artificial insemination, in vitro fertilization, gamete intrafallopian
transfer, zygote intrafallopian transfer, fertility-enhancing drugs, and
surrogate motherhood.

\subsubsection{Sexually transmitted diseases}
\label{sec:orgebe445d}
Major bacterial STDs include gonorrhea, syphilis, and chlamydia. The most
dangerous viral STD is HIV. Hepatitis BUT can be prevented by a vaccine. Genital
herpes is irritating but not particularly deadly. HPV can cause warts and is a
risk factor for cervical cancer---it, to, can be prevented by a vaccine. Yeast,
normally present in the vagina, can multiply and cause a yeast infection. Pubic
lice are tiny arthropods that are transmitted during intimate contact or by
contact with clothes or bedding. You can reduce your risk of contracting an STD
with a little effort. Choose your partner wisely, use a barrier method of birth
control, and , if you suspect you have a disease, get tested promptly.

\subsection{Ch. 8}
\label{sec:orgdb0129b}
\begin{description}
\item[{Cardiovascular system}] Provides the power to move the blood, and the
vascular system represents the network of branching conduit vessels
through which the blood flows.
\begin{itemize}
\item Greek \emph{kardia}, hear Latin \emph{vasculum} small vessel
\end{itemize}
\end{description}

\subsubsection{Blood vessels transport blood}
\label{sec:org64e77af}
A branching system of thick-walled \emph{arteries} \uline{distributes blood} to every area of
the body. \emph{Arterioles} \uline{regulate blood flow} to local regions, and precapillary
sphincters/ \uline{regulate flow} into individual \emph{capillaries}. Capillaries/ consisting
of a single layer of cells exchange materials with the \emph{interstitial fluid}. The
\emph{lymphatic system} \uline{removes excess fluid}. The thin-walled \emph{veins} \uline{return blood} to the
heart and serve as a volume reservoir for blood.

A branching network of blood vessels transports blood to all parts of the body.
The network is so extensive that if our blood vessels were laid end to end, they
would stretch 60,000 miles!

There are three major types of blood vessels: \emph{arteries}, \emph{capillaries}, and \emph{veins}.
Thick-walled arteries transport blood to body tissues under high pressure. As
blood leaves the heart it is pumped into the large, muscular arteries.

\begin{description}
\item[{Arteries}] Transport blood away from the heart. Larger arteries have a thick
muscle layer because they need to withstand high pressure from the heart.

The vessel is a sandwich of three distinct layers surrounding the \emph{lumen}, or
hollow interior of the vessel:
\begin{enumerate}
\item \emph{endothelium}.
\item Largest layer of \uline{smooth muscle} with interwoven elastic \uline{connective tissue}.
\item Outermost layer of large and medium-sized arteries have a tough supportive
layer of \uline{connective tissue}, primarily collagen.
\end{enumerate}

\item[{Aneurysm}] [ballooning of the artery wall] The \emph{endothelium} becomes damaged,
blood seeps through the injured area and the two outer layer, splitting
them apart.
\item[{Endothelium}] \emph{Squamous} \emph{epithelial cells} in a thin, flattened layer. A
\uline{continuation of} the lining of the \uline{heart}.
\item[{Capillary}] \emph{Epithelial cells} of capillary \emph{endothelium}.
\end{description}

\subsubsection{Arterioles and precapillary sphincters regulate blood flow}
\label{sec:orgbe0d665}
Eventually blood reaches the \uline{smallest} arteries, or \emph{arterioles}.

\begin{description}
\item[{Precapillary sphincter}] Smooth muscle that serves as gates for
controlling blood flow into individual capillaries.
\begin{description}
\item[{Vasoconstriction}] \emph{Arterioles} and \emph{precapillary sphincters} reduce
diameter \(\rightarrow\) reduces blood flow.
\item[{Vasodilation}] \emph{Arterioles} and \emph{precapillary sphincters} increase diameter
\(\rightarrow\) increase blood flow.
\end{description}
\item[{Arterioles}] Little \emph{arteries}. Simpler in structure than arteries because
they don't need to support as much pressure as arteries do. The
diameter is control by \emph{smooth muscle} control, the \emph{precapillary
sphincters}.
\end{description}

\subsubsection{Capillaries: Where blood exchanges substances with tissues}
\label{sec:org9edbd6c}
Porous walls allow blood to \uline{exchange} \emph{oxygen}, \emph{carbon dioxide}, \emph{nutrients}, and
waste/ products with \emph{tissue cells}.

\begin{center}
\includegraphics[width=.9\linewidth]{Ch. 8/slide_19_2016-10-25_22-44-43.jpg}
\end{center}

\begin{description}
\item[{Capillaries}] Thin-walled vessels about as think as the \emph{red blood cells}
that travel through them. Biological strainers that permit selective
\uline{exchange} of substances with the \emph{interstitial fluid}.
\item[{Capillary beds}] Found in all areas of the body. Capillary walls consist
of a single layer of \emph{squamous epithelial cells}.
\end{description}

\subsubsection{Lymphatic system helps maintain blood volume}
\label{sec:org00e5a97}
The \emph{lymphatic system} is somewhat similar to the venous system of blood
vessels, except that the fluid it contains \emph{lymph} does \uline{not} contain \emph{plasma
proteins} or \emph{red blood cells}.

\begin{description}
\item[{Lymphatic system}] Collection system of larger vessels where excess
plasma fluid is absorbed by blind-ended capillaries.
\end{description}

\subsubsection{Veins return blood to the heart}
\label{sec:org2e6559a}
Blood flows back to the heart through \emph{venules} (small veins) and \emph{veins}. Has
\uline{three layers} similar to \emph{arteries}, but much \uline{thinner}. Veins have a \uline{larger}
\emph{\uline{lumen}}, however.

Pressure goes down as it moves through the cardiovascular system, so veins
require only a small fraction of the pressure in arteries. Veins accommodate
\uline{large volumes} of blood at \uline{low pressures}.

Most veins contain valves consisting of small folds of the inner layer that
protrude into the lumen; permitting only one-way flow.

\begin{description}
\item[{Veins}] Carries blood to the heart.

Blood pools into your legs and feet when standing upright. People who
spend a lot of time on their feet may develop \emph{varicose veins}.

\item[{Vericose veins}] Permanently swollen veins that look twisted and bumpy
from pooled blood.
\end{description}

\subsubsection{The heart pumps blood through the vessels}
\label{sec:org5c9df22}
The heart wall consists of \uline{three layers}: the \emph{epicardium}, the \emph{myocardium}, and
the \emph{endocardium}. The heart contains \uline{four chambers} and \uline{four one-way valves}.
The \emph{right atrium} and \emph{right ventricle} pump blood \uline{to} the \emph{lungs}; the \emph{left
atrium} and \emph{left ventricle} pump blood \uline{to} the rest of the \emph{body}. Each \emph{cardiac
cycle} is a repetitive sequence of contraction (\emph{systole}) and relaxation
(\emph{diastole}).

The \emph{heart} is located in the \emph{thoracic cavity} \uline{between} the \emph{lungs} and behind the
sternum/, or breastbone. It consists mostly of \emph{cardiac muscles}.

\begin{description}
\item[{Cardiac muscle}] Doesn't connect to bone. \uline{Pumps} ceaselessly in a
squeezing motion to \uline{propel blood} through the \emph{blood vessels}.
\item[{Pericardium}] Tough, fibrous sac. \uline{Protects} the \emph{heart}, \uline{anchors} it to
surrounding structures, and \uline{prevents} it from \uline{overfilling} with blood.
\item[{Pericardial cavity}] Contains a film of lubricating fluid that reduces
friction and allows the \emph{heart} and the \emph{pericardium} to glide smoothly
against each other when the heart contracts.
\end{description}

There are three layers in the walls of the heart.


\begin{center}
\includegraphics[width=.9\linewidth]{Ch. 8/The-pericardial-layers-and-layers-of-the-heart-wall_2016-10-25_22-42-22.jpg}
\end{center}

\begin{description}
\item[{Epicardium}] Outermost layer consisting of a \uline{thin layer} of \emph{epithelial} and
\emph{connective tissue}.
\item[{Myocardium}] Middle layer consisting mainly of \emph{cardiac muscle} that forms
the bulk of the heart. The layer that \uline{contracts} every time the heart
beats.
\item[{Endocardium}] Innermost, thin \emph{endothelial} layer resting on a layer of
\emph{connective tissue}. Continuous with the \emph{endothelium} that lines the blood
vessels.
\end{description}

Problems*

\begin{description}
\item[{Pericarditis}] Inflammation of the \emph{pericardium}.
\end{description}

\subsubsection{The heart has four chambers and four valves}
\label{sec:org036e012}
\emph{Blood} \uline{returning} to the \emph{heart} from the body's tissues \uline{enters} the heart at the
\emph{right atrium}. From the right atrium, the blood passes through a valve \uline{into}
the \emph{right ventricle}. The right ventricle is more muscular than the right
atrium because it pumps blood at considerable pressure through a second
valve and \uline{into} the \emph{artery} leading \uline{to} the \emph{lungs}.

\emph{Blood} \uline{returning from} the \emph{lungs} \uline{to} the \emph{heart} \uline{enters} the \emph{left atrium}, then
passes through a third valve \uline{into} the \emph{left ventricle}. The very muscular left
ventricle pumps blood through a fourth valve \uline{into} the body's \uline{largest} \emph{artery},
the \emph{aorta}.

\begin{center}
\includegraphics[width=.9\linewidth]{Ch. 8/13453870-Schematic-representation-of-the-human-circulatory-system-Poster-Stock-Vector_2016-10-25_23-00-21.jpg}
\end{center}

\begin{description}
\item[{Atria}] [two atria, each an atrium] Each goes to the corresponding
ventricle. Body attaches to the right atrium. Lungs attach to the left
atrium.
\item[{Ventricles}] [two ventricles] Left ventricle goes to the body. Right
ventricle goes to the lungs.
\item[{Septum}] Muscular partition separating the right and left sides of the
heart.
\item[{Atrioventricular (AV) valves}] Between the corresponding \emph{atrium} and
\emph{ventricle} is an \emph{atrioventricular valve} which \uline{prevents} the \uline{backwards
flow} of \emph{blood} in the \emph{heart}.
\item[{Semilunar valves}] [two, pulmonary and aortic] Prevent backflow into the
ventricles from the main arteries leaving the heart when the heart
relaxes.
\end{description}

\subsubsection{The pulmonary circuit provides for gas exchange and the systemic circuit serves the rest of the body}
\label{sec:org651be9c}
\begin{description}
\item[{Pulmonary circuit}] The circuit where blood travels to and from the
lungs.
\item[{Systemic circuit}] The circuit where blood travels to and from the
body.
\item[{Coronary arteries}] \uline{Supply} the \emph{heart muscle} with \emph{blood}. They branch from
the \emph{aorta} just above the \emph{aortic semi-lunar valve} and \uline{encircle} the
\emph{heart's} surface.
\begin{itemize}
\item Latin \emph{corona} encircling like a crown
\end{itemize}
\item[{Cardiac veins}] \uline{Collect} the \emph{blood} \uline{from} the \emph{capillaries} \uline{in} the \emph{heart
muscle} and channel it back \uline{to} the \emph{right atrium}.
\end{description}

\subsubsection{The cardiac cycle: The heart contracts and relaxes}
\label{sec:orgfc5f031}
A complete \emph{cardiac cycle} involves \uline{contraction} of the two \emph{atria}, which forces
\emph{blood} \uline{into} the \emph{ventricles}, followed by \uline{contraction} of the two \emph{ventricles},
which pumps blood \uline{into} the \emph{pulmonary artery} and the \emph{aorta}, followed by a
\uline{relaxation} of the entire \emph{heart}.

\begin{center}
\includegraphics[width=.9\linewidth]{Ch. 8/figure-40-03-03_2016-10-25_23-21-01.jpeg}
\end{center}

\begin{description}
\item[{Systole}] The period of contraction.
\item[{Diastole}] The period of relaxation.
\item[{Cardiac cycle}] The entire sequence of contraction and relaxation.
\begin{enumerate}
\item \emph{Atrial systole}. \uline{Contraction} starts, the heart is \uline{already} nearly filled
with \emph{blood} that \uline{entered} the \emph{ventricles} and \emph{atria} passively during the
previous \emph{diastole}. During \emph{atrial systole}, both \emph{atria} \uline{contract}, raising
blood pressure in the \emph{atria}, then \uline{filling} the two \emph{ventricles} to
capacity.
\item \emph{Ventricular systole}. \uline{Contraction} that began in the \emph{atria} spreads to the
\emph{ventricles}, then both \emph{ventricles} \uline{contract} simultaneously. \emph{Pulmonary} and
\emph{Aortic semilunar valves} \uline{open} and \emph{blood} is \uline{ejected} into the \emph{pulmonary
trunk} and the \emph{aorta}.
\item \emph{Diastole}. Both \emph{atria} and both \emph{ventricles} are \uline{relaxed}. Pressure within
falls, the \emph{pulmonary} and \emph{aortic semilunar valves} \uline{close}, \uline{preventing
backflow}, then the \emph{AV valves} \uline{open} and blood begins to \uline{flow} passively
\uline{into} the \emph{heart}.
\end{enumerate}
\end{description}

\subsubsection{Cardiac conduction system coordinates contraction}
\label{sec:org6773634}
The coordinated sequence of the cardiac cycle is due to the \emph{cardiac
conduction system} a group of specialized \emph{cardiac muscle cells} that \uline{initiate}
and \uline{distribute} electrical \emph{impulses} throughout the \emph{heart}.

\begin{description}
\item[{Sinoatrial (SA) node}] A small mass of \emph{cardiac muscle cells} located \uline{near}
the \uline{junction} of the \emph{right atrium} and \emph{superior cardiac muscle cells}
elsewhere in the \emph{heart}.
\item[{Atrioventricular (AV) node}] Mass of muscle cells between the \emph{atria} and
\emph{ventricles} that \uline{receives signals}.
\item[{Atrioventricular (AV) bundle}] From the \emph{AV node}, the \uline{electrical signal
sweeps} to this bundle or group of conducting fibers in the \emph{septum}
between the two \emph{ventricles}.
\end{description}

\subsubsection{Blood exerts pressure against vessel walls}
\label{sec:org4ce27ba}
\emph{Blood pressure} is the force that blood exerts on the wall of a \emph{blood vessel}.
It is measured as two numbers corresponding to \emph{systolic} and \emph{diastolic
pressures}. \emph{Hypertension} (high blood pressure) is a serious risk factor for
\emph{cardiovascular disease} and other health problems.

\begin{center}
\begin{tabular}{llcl}
\textbf{Blood pressure category} & \textbf{systolic} (mm Hg) &  & \textbf{Diastolic} (mm Hg)\\
\hline
Normal & Less than 120 & and & Less than 80\\
Prehypertension & 120--139 & or & 80--89\\
Hypertension, Stage 1 & 140--159 & or & 90--99\\
Hypertension, Stage 2 & 160 or higher & or & 100 or higher\\
\end{tabular}
\end{center}

\begin{description}
\item[{Blood pressure}] \uline{Force} that \emph{blood} \uline{exerts} on the wall of a \emph{blood vessel} as
a result of the pumping action of the \emph{heart}.
\item[{Sphygmomanometer}] A tool for measuring \emph{blood pressure}.
\item[{Systolic pressure}] The \uline{highest pressure} of the cycle reached \uline{during}
\emph{ventricular systole} when the \emph{ventricles} \uline{contract} to \uline{eject} \emph{blood} from
the \emph{heart}.
\item[{Diastolic pressure}] The \uline{lower pressure} occurring during \emph{ventricular
diastole} when the \emph{ventricles} \uline{relax}.
\item[{Hypertension}] \emph{Blood pressure} higher than normal. Significant risk factor
for \emph{cardiovascular disease}, because the \uline{greater pressure} means \uline{more
strain} on \emph{cardiovascular system}.

It is called the "\(_{\text{silent}}\) killer\_" because usually it has no symptoms.
\item[{Hypotension}] When blood pressure is too low. Generally only a problem
only if \emph{blood pressure} falls enough to reduce \emph{blood} flow to the \emph{brain},
causing \uline{dizziness} and \uline{fainting}.
\end{description}

\subsubsection{Cardiovascular disorders: A major hearth issue}
\label{sec:org0c3907f}
\emph{Cardiovascular disorders} are the number one killer in the United States.
Most disorders are caused either by conditions that result in failure of the
\emph{heart} as a pump or by conditions in which \uline{damage} to \emph{blood vessels} \uline{restricts
flow} or \uline{ruptures vessels}.

\begin{description}
\item[{Heart attack}] (\emph{myocardial infarction}) Sudden death of an area of \emph{heart
tissue} due to oxygen starvation.
\item[{Congestive heart failure}] High \emph{capillary blood pressure} causes more
fluid that usual to filter out of the \emph{capillaries} and into the
\emph{interstitial space}, causing fluid congestion.
\end{description}

\subsection{Ch. 11}
\label{sec:orgf21f3f2}
\subsubsection{The Nervous System}
\label{sec:org1bd6448}
The nervous system has two major subdivisions: the \emph{central nervous system}
(CNS), consisting of the \emph{brain} and \emph{spinal cord}, and the \emph{peripheral nervous
system} (PNS), which includes all parts of the nervous system that lie
outside the CNS. The motor division of the PNS has a \emph{somatic division}, which
controls \emph{skeletal muscles}, and an \emph{autonomic division}, which controls \emph{smooth
muscles}, \emph{cardiac muscles}, and \emph{glands}.

The nervous system is the body's main \uline{control} system. Controls physical
\uline{movements}, maintains \uline{homeostasis} of many internal variables, and even
initiates our \uline{higher thought} processes and \uline{emotions}.

\begin{center}
\includegraphics[width=.9\linewidth]{./labs/Selection_004_2016-10-03_17-48-16.png}
\end{center}

\begin{description}
\item[{Central nervous system}] \{CNS\} Consists \emph{exclusively} of the \emph{brain} and
\emph{spinal cord}; all other nervous tissue is considered to be part of the
peripheral nervous system

\begin{description}
\item[{Brain}] An organ of soft nervous tissue contained in the skull.
\item[{Spinal cord}] Bundle of \emph{nervous tissue} and \emph{support cells} that
extends from the \emph{medulla oblongata} in the brain stem to the lumbar
region of the vertebral column.
\end{description}

\item[{Peripheral nervous system}] \{PNS\} All nervous tissue considered not to
be part of the brain or spinal cord.

\begin{description}
\item[{Sensory nerves}] \{afferent nerves\} Nerves that collect information
from your surrounds or from parts of your body and transmit it
\emph{toward} the CNS for processing.
\begin{itemize}
\item Signals from external environment
\item Signals from skin, tendons, and muscles
\item Signals from internal organs
\end{itemize}
\item[{Motor nerves}] \{efferent nerves\} Nerves that transmit commands \emph{away}
from the CNS to muscles, organs, glands, and vessels in response
to information collected from the sensory nerves.

Also \emph{motor neuron}, a nerve cell whose cell body is located in the
spinal cord and whose fiber (\emph{axon}) projects outside the \emph{spinal
cord} to directly or indirectly control effector organs. They are
\emph{efferent nerve fibers}.
\item[{Somatic nervous system}] Controls voluntary and involuntary skeletal
muscle movement.
\item[{Autonomic nervous system}] Made up of involuntary pathways that
control \emph{smooth muscle}, \emph{cardiac muscle}, and \emph{glands}. Further divides
into two systems\dots{}
\begin{description}
\item[{Sympathetic}] [flight or flight] Excitatory pathway that
helps your body in times of \uline{stress} by \uline{elevating} blood
pressure, heart rate, respiratory rate, and the volume of
blood flow to muscles.
\item[{Parasympathetic}] [rest and digest] System most active in tames
of relaxation, and it stimulates gland secretions and
digestive processes while dampening sympathetic effects.
\end{description}
\end{description}

\item[{Neuroglia}] Specialized connective tissue cells of the nervous system.
They do not generate or transmit impulses. Some secrete a special
substance called \emph{myelin}. Neuroglial cells \uline{support} and \uline{protect} \emph{neurons}.
Neuroglial cells called \emph{Schwann cells} (in the PNS) and \emph{oligodendrocytes}
(in the CNS) form \emph{myelin sheaths} that \uline{protect} \emph{axons} and \uline{speed
transmission} of impulses.

In the PNS, many \emph{neuron axons} are enclosed and protected by \emph{neuroglial
cells} called \emph{Schwann cells}. In the CNS, they are oligodendrocytes.

Greek // neuron, glue
\begin{description}
\item[{Schwann cells}] Neuroglia that secrete \emph{myelin} in the \emph{peripheral
nervous system}. Schwann cells wrap themselves around a short
segment of an axon many times as a sort of insulating blanket,
creating a \uline{shiny white protective layer} around the axon called a
\emph{myelin sheath}.

Principal \emph{glia} of the peripheral nervous system
\item[{Oligodendrcytes}] Neuroglia that secrete myelin in the \emph{central
nervous system}. Unlike the sheath formed by Schwann cells, the
sheath formed by oligodendrocytes degenerates once the axon it
protects is destroyed, so the neurons of the central nervous
system do not regenerate after injury--this is why spinal cord
injuries and CNS disorders result in a permanent change or loss of
function.
\begin{itemize}
\item Do Oligodendrocytes send nerve signals? \textbf{NO}, no \emph{neuroglia} send
signals.
\end{itemize}
\item[{Astrocytes}] neuroglial cells that form barriers around the neurons,
shielding them from direct contact with substances carried in the
blood. Form the basis of the blood-brain barrier.
\end{description}

\item[{Myelin sheath}] Surrounds the \emph{axon} of some nerve cells forming an
electrically insulating layer around the neuron. White in color.

Three important functions: \textbf{1)} It saves the \emph{neuron} energy, \textbf{2)} speeds up
the transmission of impulses by preventing nearly all leakage of
charged ions across the axon membrane causing action potentials to jump
between nodes of Ranvier causing the \emph{saltatory conduction}, and \textbf{3)} it
helps damaged or severed axons of the peripheral nervous system
regenerate.
\end{description}

\begin{center}
\includegraphics[width=.9\linewidth]{book/AfmsJ_2016-10-25_20-22-30.jpg}
\end{center}

\begin{description}
\item[{Nerve}] 


\item[{Neurons}] Cell present in the nervous system highly specialized for
communication.

Carry electrical impulses in nervous tissue. Myelinated neurons can
transmit electrical impulses more quickly than unmyelinated neurons.

Also \emph{interneurons} within the CNS transmit impulses between components
of the CNS. \emph{Interneurons} input from sensory neurons, integrate this
information, and influence the function of other neurons.

\begin{description}
\item[{Axon}] long, slender projection of a nerve cell, or \emph{neuron}, that
typically conducts electrical impulses away from the neuron's cell
body. Contains a small amount of \emph{cytoplasm}. Each axon terminal ends
in a small, rounded tip called an \emph{axon bulb}.
\item[{Dendrite}] Branched projections of a \emph{neuron} that act to propagate the
electrochemical stimulation received from other neural cells to the
cell body, or \emph{soma}.
\begin{itemize}
\item Greek \emph{déndron} tree
\end{itemize}
\item[{Soma}] Bulbous end of a \emph{neuron} containing the \emph{cell nucleus}. The
survival of some sensory neurons depends on \emph{axon terminals} making
contact with sources of survival factors that prevent apoptosis.
\begin{itemize}
\item Greek \emph{somas} body
\end{itemize}
\item[{Apoptosis}] process of programmed cell death that occurs in
multicellular organisms. Between 50 and 70 billion cells die each day
due to apoptosis in the average human adult.
\begin{itemize}
\item Ancient Greek \emph{apoptosis} a falling off
\end{itemize}
\item[{Node of Ranvier}] [myelin sheath gaps] Periodic gaps in the insulating
\emph{myelin sheath} of myelinated axons where the axonal membrane is
exposed to the extracellular space.
\end{description}
\end{description}

\subsubsection{Neurons initiate action potentials}
\label{sec:org1ce0237}
There are \uline{millions} of these \emph{Sodium} and \emph{Potassium} gates. Neurons generate
and transmit electrical impulses from one part of the body to another.
Sensory neurons transmit impulses to the CNS. Interneurons transmit impulses
between components of the CNS. Motor neurons transmit impulses away from the
CNS to muscles and glands.

Action potential is greater in larger-diameter axons and in axons covered by
an \emph{insulating sheath}. A neuron's \emph{resting potential} of about -70 millivolts
is \uline{maintained} by the \uline{constant action} of the sodium-potassium pump. Impulses
arriving from other neurons can cause small, local changes in the neuron's
membrane potential called \emph{graded potentials}. The sum of all graded
potentials may initiate a self-propagating, all-or-none action potential in
a neuron. An action potential involves \emph{three events}: \uline{depolarization},
\uline{repolarization}, and \uline{reestablishment}.

\begin{description}
\item[{Depolarization}] \emph{Sodium} moves into the axon. When threshold is exceed,
voltage-sensitive Na\(^{\texttt{+}}\) channels in the axon's membrane open briefly
and Na\(^{\texttt{+}}\) ions diffuse rapidly \emph{into} the cytoplasm of the axon.
\item[{Repolarization}] \emph{Potassium} moves out of the axon. After a short delay,
Na\(^{\texttt{+}}\) channels close automatically. But the reversal of the membrane
polarity triggers the opening of K\(^{\texttt{+}}\) channels that allows K\(^{\texttt{+}}\)
ions to diffuse rapidly \emph{out} of the cell.
\item[{Reestablishment}] K\(^{\texttt{+}}\) channels are slow to close, so there is a brief
overshoot of membrane voltage during which the interior of the \emph{axon} is
slightly hyperpolarized. Shortly after the K\(^{\texttt{+}}\) channels close, the
resting potential is reestablished.
\end{description}

\subsubsection{Information is transferred from a neuron to its target}
\label{sec:org4d7d169}
At the axon terminal of a neuron, the action potential and its information
must be converted to another form for transmittal to its target (muscles
cell, gland cell, or another neuron). It reaches a junction called a
\emph{synapse}, where a chemical is released called a \emph{neurotransmitter} because it
transmits a signal from a neuron to its target.

\begin{description}
\item[{synapse}] A junction where information is converted from one form to
another. 
\begin{description}
\item[{presynaptic membrane}] The cell membrane of the neuron that is sending
the information.
\item[{postsynaptic membrane}] The membrane of the cell that is about to
receive the information.
\item[{synaptic cleft}] Fluid-filled gap that separates the pre- and post-
synaptic membranes.
\end{description}

\item[{neurotransmitter}] 
\end{description}

\subsubsection{regions of the brain}
\label{sec:org720bed0}
\begin{description}
\item[{Parietal lobe}] Interprets sensory information from skin.
\item[{Occipital lobe}] Processes visual information.
\item[{Frontal lobe}] Initiates motor activity. Responsible for speech.
Conscious thought.
\item[{Temporal lobe}] Interprets auditory information. Comprehends language.
Perceptual judgment.
\begin{description}
\item[{Prefrontal cortex (PFC)}] The \emph{cerebral cortex} which covers the front
part of the \emph{frontal lobe}. Implicated in planning complex cognitive
behavior, personality expression, decision making, and moderating
social behavior.
\end{description}
\end{description}

\subsubsection{parts of the brain}
\label{sec:org88ed066}
\begin{description}
\item[{Pituitary gland}] Referred to as the 'master gland', but is about the
size of a pea. Protrusion off the bottom of the \emph{hypothalamus} It
controls several of the other \emph{hormone glands} (e.g. \emph{adrenals}, \emph{thyroid}).
It sits in a bony hollow called the \emph{pituitary fossa} (it was called
\emph{sella turcica} elsewhere as well).
\item[{amygdala}] A part of the \emph{limbic system} in the \emph{temporal lobe}. When the
amygdala senses threat, it becomes over-activated, blocking new
sensory information from accessing the memory and association
circuits.

THIS MEANS THAT IF A STUDENT IS STRESSED OUT, THEY ACTUALLY CAN NOT
LEARN. THE AMYGDALA IS BLOCKING information from going to higher
cognitive centers of the brain, like the \emph{prefrontal cortex} where
information is processed, associated, and stored for later retrieval
and executive functioning
\end{description}

\begin{enumerate}
\item \textbf{Forebrain}
\label{sec:org9995983}
Determines most \uline{complex behavior}, including \uline{emotions} and conscious though.
Important to the forebrain is \emph{hypothalamus} and \emph{thalamus}, \emph{limbic system}, and
\emph{cerebrum}.

The hypothalamus and thalamus \uline{maintain homeostasis} and \uline{process information}.
The hypothalamus is a small region at the base of the forebrain that
coordinates some automatic functions of the \emph{pituitary gland}.

\begin{description}
\item[{Cerebrum}] Coordinates \uline{language}. Controls \uline{decision making}. Produces
\uline{conscious thought}.
\item[{Corpus callosum}] Bridges the two cerebral hemispheres.
\item[{Thalamus}] Receives, processes, and transfers information.
\item[{Hypothalamus}] Part of the \emph{limbic system}. Works mostly with the
\emph{autonomic nervous system}.
\end{description}

\item Midbrain
\label{sec:org6afda63}
Most function of the midbrain relate to vision and hearing. Visual and
auditory sensory inputs pass through the midbrain before being relayed to
higher brain centers.

\begin{description}
\item[{Midbrain}] Relays visual and auditory inputs. Coordinates movement.
\end{description}

\item \textbf{Hindbrain}
\label{sec:orgdab2217}
The evolutionarily oldest, more primitive hindbrain controls must automatic
activities.

\begin{description}
\item[{Medulla oblongata}] Controls automatic functions of internal organs.
\item[{Pons}] Connects cerebellum, spinal cord with higher brain centers.
Aids medulla in regulating respiration.
\item[{Cerebellum}] Controls basic and skilled movements.
\end{description}
\end{enumerate}

\subsubsection{misc vocabulary}
\label{sec:org08c5946}
\begin{description}
\item[{saltatory conduction}] Leaping pattern of conduction along myelinated
neurons.
\begin{itemize}
\item Latin \emph{saltare} dance
\end{itemize}
\item[{Neuroplasticity}] Plasticity is the capacity of the brain to change with
learning. There are three circumstances where neuroplasticity occurs in
the brain:
\begin{enumerate}
\item At the beginning of life: when the immature brain organizes itself.
\item In case of brain injury: to compensate for lost functions or maximize
remaining functions.
\item Through adulthood: whenever something new is \uline{learned} and \uline{memorized}.
\end{enumerate}
\end{description}

\subsubsection{Common neurotransmitters}
\label{sec:orgaa34f73}
\emph{NTS} stands for Neurotransmitter Senses. Chemical messengers, are
endogenous chemicals that enable neurotransmission. 

Neurotransmitters do not use a transport system.

\begin{description}
\item[{Neurohormone}] Just a hormone.

\item[{Acetylcholine}] Excitatory on skeletal muscles; excitatory or inhibitory
at other sites, depending on receptors. Neurotransmitter used at the
\emph{neuromuscular junction}---it is the \uline{chemical} that \emph{motor neurons} of the
nervous system release in order to \uline{activate muscles}.
\begin{itemize}
\item \emph{Neuromuscular junctions}, \emph{autonomic nervous systems}, \emph{brain}
\item binds muscle cell(s)---causes depolarization.
\end{itemize}
\item[{Norepinephrine}] [noradrenaline] Excitatory or inhibitory, depending on
receptors, plays a role in emotions. Synthesized and released by
\emph{central nervous system} and also by the \emph{sympathetic nervous system}.
Norepinephrine is \uline{produced} in \uline{closely packed} \emph{brain} \emph{cell neurons}. Exert
powerful effect on the brain. Sympathetic ganglia, located in the
\emph{spinal cord} or in the \emph{abdomen}, uses norepinephrine; norepinephrine also
is released directly into the bloodstream by the \emph{adrenal glands}.
Mobilizes the brain and body for action.
\begin{itemize}
\item Latin \emph{noradrenaline} at or alongside the kidneys
\item Areas of brain and spinal cord, autonomic nervous system
\end{itemize}
\item[{Serotonin}] Usually inhibitory; involved in moods, sleep cycle,
appetite
\begin{itemize}
\item Areas of brain, spinal
\end{itemize}
\item[{Dopamine}] Excitatory or inhibitory, depending on receptors; plays a
role in emotions
\begin{itemize}
\item Areas of brain, parts of peripheral nervous systems
\end{itemize}
\item[{Glutamate}] Usually excitatory; major excitatory neurotransmitter in
brain. By a wide margin, \emph{glutamate} is abundant neurotransmitter in the
vertebrate nervous system. well over 90\% of the synaptic connections in
the human brain happen with \emph{glutamate}.
\begin{itemize}
\item Areas of brain, spinal cord
\end{itemize}
\item[{Endorphins}] Natural opiates that inhibit pain; usually inhibitory
\begin{itemize}
\item Many areas in brain, spinal cord
\end{itemize}
\item[{Gamma-aminobutyric acid}] Usually inhibitory; principal inhibitory
neurotransmitter in brain
\begin{itemize}
\item Areas of brain, spinal cord
\end{itemize}
\item[{Somatostatin}] Usually inhibitory; inhibits pancreatic release of growth
hormone
\begin{itemize}
\item Areas of brain, pancreas
\end{itemize}
\end{description}

\section{other}
\label{sec:org4219286}
\subsection{Medical Professions}
\label{sec:orga452b23}

\begin{description}
\item[{Cardiology}] Medical science dealing with the heart and heart diseases.
\begin{itemize}
\item Latinized Greek \emph{cardia, kardia} "heart"
\end{itemize}

\item[{Dermatology}] Study of the skin and its diseases.
\begin{itemize}
\item Latin \emph{derm} "skin"
\end{itemize}

\item[{Endocrinology}] Study of the hormones, hormone-secreting glands, and their
diseases.
\begin{itemize}
\item Latin \emph{endo} "within"
\end{itemize}

\item[{Epidemiology}] Study of the factors determining the distribution and
frequency of the occurrence of health-related conditions within a
defined human population.
\begin{itemize}
\item Latin \emph{epi} "upon, among" \emph{demos} "people"
\end{itemize}

\item[{Gastroenterology}] Study of the stomach and intestines and their diseases.
\begin{itemize}
\item Latin \emph{gastro} "stomach" \emph{entero} "intestines"
\end{itemize}

\item[{Geriatrics}] Branch of medicine dealing with older individuals and their
medical problems.
\begin{itemize}
\item Latin \emph{geros} "old"
\end{itemize}

\item[{Gynecology}] Study of the female reproductive system and its diseases.
\begin{itemize}
\item Latin \emph{gyno} "pertaining to women"
\end{itemize}

\item[{Hematology}] Study of the blood and blood diseases.
\begin{itemize}
\item Latin \emph{hem} "blood"; also \emph{hemo} and \emph{hemato}
\end{itemize}

\item[{Histology}] Study of the structure and function of tissues, also called
microscopic anatomy.
\begin{itemize}
\item Latin \emph{hist} "tissue"
\end{itemize}

\item[{Immunology}] Study of the body's resistance to disease.
\begin{itemize}
\item Latin \emph{im-} "negation"
\end{itemize}

\item[{Neonatology}] Study of newborns. 
\begin{itemize}
\item Latin \emph{neo} "new"
\end{itemize}

\item[{Neurology}] Study of the nervous system and its disorders.
\begin{itemize}
\item Latin \emph{neuro} "nerve"
\end{itemize}

\item[{Obstetrics}] Branch of medicine dealing with pregnancy and childbirth.
\begin{itemize}
\item Latin \emph{obstare} "to stand by"
\end{itemize}

\item[{Oncology}] Study of cancers
\begin{itemize}
\item Latin \emph{onco} "cancer"
\end{itemize}

\item[{Ophthalmology}] Study of the eye and eye diseases
\begin{itemize}
\item Latin \emph{onco} "cancer"
\end{itemize}

\item[{Orthopedics}] Branch of medicine dealing with the muscular and skeletal
systems and their problems.
\begin{itemize}
\item Greek \emph{orthos} "correct"
\end{itemize}

\item[{Otolaryngology}] Study of the ear, nose, and throat (larynx), and their
diseases.
\begin{itemize}
\item Latin \emph{ot} "ear" \emph{lary} "larynx"
\end{itemize}

\item[{Pathology}] Study of structural and functional changes that disease
produces.
\begin{itemize}
\item Greek \emph{pathos} "suffering, disease"
\end{itemize}

\item[{Pharmacology}] Study of drugs and their uses in the treatment of
disease.
\begin{itemize}
\item Latin \emph{pharm} "drug"
\end{itemize}

\item[{Psychiatry}] Branch of medicine dealing with the mind and its disorders.
\begin{itemize}
\item Greek \emph{psyc} "the mind"
\end{itemize}

\item[{Radiology}] Study of X-rays and radioactive substances and their uses in
the diagnosis and treatment of diseases.
\begin{itemize}
\item Latin \emph{radio} "wireless transmission"
\end{itemize}

\item[{Toxicology}] Study of poisonous substances and their effects upon body
parts.
\begin{itemize}
\item Latin \emph{toxi} "poison"
\end{itemize}

\item[{Urology}] Branch of medicine dealing with the urinary system, apart from
the kidneys (nephrology), and male reproductive system, and their
diseases.
\begin{itemize}
\item Latin \emph{uro} "urine"
\end{itemize}
\end{description}
\subsection{lab}
\label{sec:org084b1d6}
\subsubsection{vocabulary (lab 2)}
\label{sec:orgc500aa2}

\begin{description}
\item[{concentration gradient}] relative difference in concentration

\item[{crenation}] [process] shriveling of the cell or the movement of water out of
the cell

\item[{diffusion}] [process] a \emph{solutes} move from an area of higher to an area of
lower concentration

\item[{equilibrium}] concentration of salt on both sides of the membrane is the same

\item[{hypertonic}] solution has greater \emph{solute} concentration than another
solution

\item[{hypotonic}] solution has lesser \emph{solute} concentration than another solution

\item[{isotonic}] solution have equal number of total molecules suspended

\item[{lyse}] [process] rupturing of the cell or the movement of water into the
cell

\item[{osmosis}] [process] \emph{solvent} molecules travel down their concentration
gradient

\item[{solution}] contains two parts, the \emph{solvent} and the \emph{solute}

\begin{description}
\item[{solvent}] dissolving agent of a solution

\item[{solute}] the molecules suspended in the solvent
\end{description}
\end{description}

\subsubsection{lab vocabulary (lab 3)}
\label{sec:org96a3135}

\begin{description}
\item[{genome}] complete copy of information

\item[{protease}] an enzyme that breaks down proteins and peptides

\item[{lysis}] the disintegration of a cell by rupture of the cell wall or
membrane

\item[{gene}] specific stretch of DNA transcribed into messenger RNA \emph{mRNA}

\item[{DNA}] Deoxyribonucleic acid

\item[{RNA}] Ribonucleic acid

\begin{description}
\item[{mRNA}] messenger Ribonucleic acid

Cellular organisms use mRNA to convey genetic information (using G,
U,A, and C to denote the nitrogenous bases guanine, uracil, adenine,
and cytosine)

\item[{tRNA}] transfer RNA

Deliver amino acids to the ribosome, where ribosomal RNA (rRNA) then
links amino acids together to form protein pairs.
\end{description}

\item[{cardiomyocyte}] Shorten and lengthen their size to proper form during the
beating of the heart.

Make up the \emph{atria} where blood enters the heart.

\begin{description}
\item[{(phase one) systole}] Rest phase, considered polarized.

\item[{(phase two) diastole}] 
\end{description}
\end{description}


\begin{description}
\item[{genotype}] Genetic constitution of an individual organism.

A listing of \emph{alleles}.

\item[{dominant (vs. recessive)}] Always expressed when present in individual's
genome. \emph{Recessive alleles} get masked by \emph{dominant alleles}.

\item[{transcription}] [process] make a complementary strand of mRNA from
genomic \emph{DNA}.

\item[{phenotype}] Results of the interaction of its genotype with the
environment. Observable.

\item[{homozygous}] Both individual's \emph{alleles} for a specific trait are the
\textbf{same}.

\item[{heterozygous}] Both individual's \emph{alleles} for a specific trait are
\textbf{different}.

\item[{codominance}] \emph{Alleles} don't encode A or B. Both alleles are expressed
equally. Observable in individuals with the blood type AB.

\item[{incomplete dominance}] Can occur with \emph{heterozygous phenotypes}.

This is the red flower and a white flower make a pink flower type of
thing.

\item[{translation}] [process] \emph{ribosomes} build proteins out of their sub-unit
pieces called \emph{amino acids}.

\item[{allele (vs. gene)}] Alternative versions of a \emph{gene}. One has two
potentially different copies of each gene.
\item[{antibody (vs. antigen)}] \emph{Protein} displayed on the surface of the cell.
RBCs created from DNA that encodes the type A allele which each display
the type A antigen, or protein on the outside of the cell membrane.
Type B cells display type B surface protein.
\item[{\emph{case} OO}] This person has inherited no genes for surface antigens and
thus has no surface antigens displayed on the surface of their RBCs.
\item[{Transcription}] 

\item[{Carbohydrate}] \emph{polysaccharides} long chains of sugars. Provide the body
with a source of fuel and energy
\begin{description}
\item[{Lipid}] Long hydrocarbon chains with a large amount of energy, energy
storage molecules. Energy storage mechanic of the body.
\item[{Glycolipids}] Contains oligosaccharides with 1-15 saccharide residues.
\item[{Phospholipids}] Contain positively charged head linked to the
negatively charged phosphate groups.
\item[{Serols}] head contain a steroid ring.
\item[{Protein}] Many different functions like, structural support, help in
body movement, defense against germs and infections.
\item[{Monosaccharides}] simple sugars composed of 3-7 carbon atoms.
Monosaccharides and disaccharides are sweet, crystalline and water
soluble substances.
\item[{glycosidic bonds}] 

\item[{free aldehye}] \emph{reducing sugars} reducing agents
\item[{free ketone}] \emph{reducing sugars} reducing agents
\end{description}
\item[{Nucleic Acid}] Large biomolecules essential for all know forms of life.
They function in encoding, transmitting, and expressing genetic
information.
\item[{Sugar}] Sweet, short-chain \emph{carbohydrates}, many of which are used in
food.
\end{description}
\subsubsection{vocabulary (lab 4)}
\label{sec:orgd88857e}

\begin{description}
\item[{theory}] Refers to a broad theme of concepts that unify a single topic.

\item[{natural selection}] Organisms that are better adapted for their
environment will reproduce more successfully.

\item[{evolution}] Better adapted organisms create a higher distributions of
allele frequencies in a population over time.

\item[{fitness}] An individual's ability to reproduce.
\end{description}

\begin{enumerate}
\item data
\label{sec:org6a188a1}

\begin{description}
\item[{replicate}] Snapshot of data that shouldn't be changed, but rather
adapted into more readable format for other things
\end{description}
\end{enumerate}
\subsection{important}
\label{sec:orgfcd8c1c}
\subsubsection{cells}
\label{sec:org516f931}

\begin{description}
\item[{Eukaryote}] Have membrane-bound organelles, especially the nucleus, which
contains the genetic material and is enclosed by the \emph{nuclear
envelope}.

\item[{Prokaryote}] A single-celled organism that lacks a membrane-bound
nucleus, mitochondria, or any other membrane-bound organelle.

All the intracellular water-soluble components (proteins, DNA, and
metabolites) are located together in the cytoplasm enclose by the cell
membrane.

\begin{description}
\item[{cytoplasm}] Material or protoplasm within a living cell, excluding the
cell nucleus.
\end{description}

\item[{Nucleus}] Contain most of the cell's genetic material, organized as
multiple long linear DNA molecules in complex with a large variety of
proteins like \emph{histones} and \emph{chromosomes}.

\begin{itemize}
\item Latin \emph{nucleus} kernel, seed
\end{itemize}

\item[{Nucleolus}] Largest structure in the nucleus of eukaryotic cells,
primarily serving as the site of ribosome synthesis and assembly.

\item[{Ribosomes}] Complex molecular machine, serving as the site of biological
protein synthesis (translation). Link amino acids together in the order
specified by mRNA molecules.

\item[{Endoplasmic reticulum}] Organelle in eukaryotic cells. Forms
interconnected network of flattened, membrane-enclosed sacs or
tube-like structures known as \emph{cisternae}. 

Folds protein molecules in sacs called \emph{cisternae}. Transport
synthesized proteins in \emph{vesicles} to the \emph{Golgi apparatus}.

\item[{Golgi apparatus}] Organelle part of the cellular endomembrane system.
Packages proteins into membrane-bound vesicles inside the cell before
being sent to destination.

\item[{Various vesicles}] Small structure within a cell; a fluid enclosed by a
\emph{lipid bilayer}. Used to carry things (?)

\item[{Mitochondria}] The powerhouse of the cell. Generate most of the cell's
adenosine triphosphate (ATP).

\item[{Cytoskeleton}] Complex network of interlinking filaments and tubules that
extend throughout the cytoplasm. The function of this depends highly on
the cell in question.

\item[{Cilia and flagella}] Projections from the cell. Move the cell itself or
move substances over or around the cell.
\end{description}

\subsubsection{Blood typing}
\label{sec:org1c25245}

\begin{center}
\includegraphics[width=.9\linewidth]{labs/tut3_compatible_blood_2016-09-20_13-13-59.jpg}
\end{center}

\begin{description}
\item[{agglutination}] mixtures have clumping

\begin{itemize}
\item Blood agglutinating means the bloods mixed are not compatible with the
kind of antibody.

\item No agglutination in tube with A antibodies means patient's RBC do not
have A antigens.

\item Agglutination in tube with B antibodies means patient's RBC have B
antigens.

\item No agglutination in the Rh tube means the patient's RBC do not have Rh
antigens, so the blood is Rh-.
\end{itemize}

\item[{ABO blood system}] \begin{description}
\item[{O}] receives O

\item[{A}] receives A and O

\item[{B}] receives B and O

\item[{AB}] receives AB, A, B, and O
\end{description}

\item[{Rh blood system}] \begin{description}
\item[{Rh+}] receives Rh+

\item[{Rh-}] receives Rh-
\end{description}

\item[{O Rh-}] \emph{universal donor}
\end{description}

\subsubsection{Anatomy and Levels of Organization}
\label{sec:org665ad4f}
\begin{enumerate}
\item Explanation of levels
\label{sec:org09079e1}

\begin{enumerate}
\item Atom

Smallest unit of an element of matter.

\item Molecule 

More than one atom in a stable association.

\item Cell

Smallest Unit of Life.

\item Tissue

An association of \emph{cells} with the same general structure and function.

\begin{enumerate}
\item Epithelium :: animal tissue that lines the cavities and surfaces of blood
vessels and organs throughout the body.

\begin{itemize}
\item Latin \emph{epi-} "on" \emph{thele} "nipple"
\end{itemize}
\end{enumerate}

\item \hyperref[sec:org8cc25d2]{Organs \& Organ Systems}

An ossociation of several tissue types that carry out a specific
function. Two more organs work together to carry out a general function
in an \emph{organ system}.

\item Organism 

An individual living being composed of several organs or organ systems.

\item Population

A group of individuals of the same species living in the same area.

\item Community

Several populations of different species who inhabit the same area and
interact with each other.

\item Ecosystem

All of the organisms in a given area plus all of the nonliving matter and
energy.

\item Biosphere

All ecosystems combined. The portion of Earth occupied by living organisms,
plus those organisms.
\end{enumerate}

\item Organs \& Organ Systems
\label{sec:org8cc25d2}

An association of several tissue types that carry out a specific function.
Two more organs work together to carry out a general function in an \emph{organ
system}.

\begin{description}
\item[{integumentary}] skin, hair, nails
\begin{description}
\item[{integument}] natural covering, as a skin, shell, or rind
\begin{itemize}
\item Latin \emph{integumentum} a covering; from \emph{integere} to cover
\end{itemize}
\end{description}
\item[{skeletal}] bones, as well as ligaments, and cartilages
\begin{itemize}
\item modern Latin \emph{sceleton} bones, bony framework of a the body; greek
\emph{skeleton soma} dried-up body, mummy, skeleton from \emph{skeletos} dried-up
or \emph{skellein} dry up, make dry, parch; PIE root \emph{skele-} to parch, wither
\end{itemize}
\item[{muscular}] muscles
\begin{itemize}
\item French \emph{muscle} muscle; Latin \emph{musculus} a muscle, literally little
mouse, diminutive of \emph{mus} mouse
\end{itemize}
\item[{nervous}] brain, spinal cords, nerves, sense organs
\begin{itemize}
\item Latin \emph{nervosus} sinewy, vigourous \emph{nervus} sinew, nerve
\end{itemize}
\item[{endocrine}] endocrine glands and hormones, pituitary gland

collection of glands of an organism that secrete hormones to the
circulatory system

\begin{itemize}
\item Latinized Greek \emph{endo}; Greek \emph{krinein} to separate, distinguish
\end{itemize}

\item[{digestive}] mouth, tongue, teeth, salivary glands, pharynx, esophagus,
stomach, liver, gallbladder, pancreas, small intestine, and large
intestine

\begin{itemize}
\item Old French \emph{digestif}; Latin \emph{digestivus} pertaining to digestion Latin
\emph{digerere} digest
\end{itemize}

\item[{cardiovascular}] heart, arteries, veins, capillaries, and blood

\begin{itemize}
\item modern Latin \emph{vascularis} of or pertaining to vessels or tubes; Latin
\emph{vasculum} a smell vessel; \emph{vas} [diminutive] vessel; Latinized Greek
\emph{cardia, kardia} heart
\end{itemize}

\item[{lymphatic}] lymphatic vessels, lymph fluid, lymph nodes, thymus gland,
and spleen are all part of the circulatory system; vital part of immune
system; carrying clear fluid called \emph{lymph} directly towards the heart.

\begin{itemize}
\item Latin \emph{lymphaticus} from \emph{lympha} water (or mad, frenzied)
\end{itemize}

\item[{respiratory}] nasal cavity, pharynx, larynx, trachea, bronchi, and lungs

\begin{itemize}
\item modern Latin \emph{respiratorius} French \emph{respiratoire} Latin \emph{respirationem}
breathing, respiration
\end{itemize}

\item[{urinary}] kidneys, ureters, urinary bladder, and urethra

\begin{itemize}
\item modern Latin \emph{urinarius}; Latin \emph{urina} urine
\end{itemize}

\item[{reproductive (male)}] scrotum, testes, prostate gland, penis, and urethra

\begin{itemize}
\item French \emph{reproduire}; Old French, Latin \emph{re-} again, back, anew, against;
Latin \emph{producere} lead or bring forth, draw out
\end{itemize}

\item[{reproductive (female)}] ovaries, uterine tubes, uterus, vagina, and clitoris

\begin{itemize}
\item French \emph{reproduire}; Old French, Latin \emph{re-} again, back, anew, against;
Latin \emph{producere} lead or bring forth, draw out
\end{itemize}
\end{description}

\item Issues and controversies
\label{sec:org8c06fa5}

The slide makes a point that each of these has a corresponding level of
problem which needs to be dealt with as well. For example, on a community we
need to consider problems about animal life that we impact, or on a cellular
level we need to deal with cloning adult animals, plants, and humans from a
single cell.
\end{enumerate}
\end{document}